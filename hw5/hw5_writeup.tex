%%%%%latex preamble%%%%%
\documentclass[titlepage]{article}\usepackage[]{graphicx}\usepackage[]{color}
%% maxwidth is the original width if it is less than linewidth
%% otherwise use linewidth (to make sure the graphics do not exceed the margin)
\makeatletter
\def\maxwidth{ %
  \ifdim\Gin@nat@width>\linewidth
  \linewidth
  \else
  \Gin@nat@width
  \fi
}
\makeatother


\usepackage{listings}
\definecolor{mygreen}{rgb}{0,0.6,0}
\definecolor{mygray}{rgb}{0.5,0.5,0.5}
\definecolor{mymauve}{rgb}{0.58,0,0.82}
\lstset{ %
  backgroundcolor=\color{white},   % choose the background color; you must add \usepackage{color} or \usepackage{xcolor}
  basicstyle=\footnotesize,        % the size of the fonts that are used for the code
  breakatwhitespace=false,         % sets if automatic breaks should only happen at whitespace
  breaklines=true,                 % sets automatic line breaking
  captionpos=b,                    % sets the caption-position to bottom
  commentstyle=\color{mygreen},    % comment style
  deletekeywords={...},            % if you want to delete keywords from the given language
  escapeinside={\%*}{*)},          % if you want to add LaTeX within your code
  extendedchars=true,              % lets you use non-ASCII characters; for 8-bits encodings only, does not work with UTF-8
  frame=single,                    % adds a frame around the code
  keepspaces=true,                 % keeps spaces in text, useful for keeping indentation of code (possibly needs columns=flexible)
  keywordstyle=\color{blue},       % keyword style
  language=Python,                 % the language of the code
  morekeywords={*,...},            % if you want to add more keywords to the set
  numbers=left,                    % where to put the line-numbers; possible values are (none, left, right)
  numbersep=5pt,                   % how far the line-numbers are from the code
  numberstyle=\tiny\color{mygray}, % the style that is used for the line-numbers
  rulecolor=\color{black},         % if not set, the frame-color may be changed on line-breaks within not-black text (e.g. comments (green here))
  showspaces=false,                % show spaces everywhere adding particular underscores; it overrides 'showstringspaces'
  showstringspaces=false,          % underline spaces within strings only
  showtabs=false,                  % show tabs within strings adding particular underscores
  stepnumber=2,                    % the step between two line-numbers. If it's 1, each line will be numbered
  stringstyle=\color{mymauve},     % string literal style
  tabsize=2,                       % sets default tabsize to 2 spaces
  title=\lstname                   % show the filename of files included with \lstinputlisting; also try caption instead of title
}
\usepackage{alltt}
\usepackage[sc]{mathpazo}
\usepackage{amsmath, amsthm, amssymb}
\usepackage{graphicx}
\usepackage[T1]{fontenc}
\usepackage{geometry}
\geometry{verbose,tmargin=2.5cm,bmargin=2.5cm,lmargin=1.5cm,rmargin=1.5cm}
\setcounter{secnumdepth}{2}
\setcounter{tocdepth}{2}
\usepackage{url}
\usepackage[unicode=true,pdfusetitle,
  bookmarks=true,bookmarksnumbered=true,bookmarksopen=true,bookmarksopenlevel=2,
breaklinks=false,pdfborder={0 0 1},backref=false,colorlinks=false]
{hyperref}
\hypersetup{pdfstartview={XYZ null null 1}}
\usepackage{float}
\usepackage{bm}
\usepackage{tikz}
 %changes default sectioning commands -> 1,a, etc.
%\usepackage{breakurl}
\renewcommand{\thesubsection}{(\alph{subsection})}
\renewcommand{\thesubsubsection}{\roman{subsection}.}
\usepackage{lastpage}
\usepackage{fancyhdr}
\pagestyle{fancy}

%%% Header and Footer %%% 
\lhead{}
\chead{\leftmark}
\rhead{}
\lfoot{Aaron Gonzales; Algorithms}
\cfoot{Homework X}
\rfoot{Page \thepage\ of \pageref{LastPage}}
\IfFileExists{upquote.sty}{\usepackage{upquote}}{}

\begin{document}

\title{Homework X, CS561, Fall 2014}
\author{Aaron Gonzales}
\maketitle


%%%% useful align for this
\section{}
\begin{quote}
  \textbf{}
\end{quote}

\begin{itemize}
  \item \textbf{}
  \item \textbf{ }
\end{itemize}

\begin{enumerate}

	\item (Probability) Solve Problem 5 on the midterm from 2013 at\\
		http://www.cs.unm.edu/$\sim$saia/classes/561-f13/mid.pdf

	\item Problem 17-2 (Making Binary Search Dynamic)

	\item Problem 22-4 (Reachability) \footnote{The answer to this problem
			can be used in an efficient randomized algorithm for estimating the
			*number* of vertices that are reachable - we may see this later in
		this class.}

	\item Professor Curly conjectures that if we do union by rank,
		\emph{without path compression}, the amortized cost of all operations
		is $o(\log n)$.  Prove him wrong by showing that if we do union by
		rank without path compression, there can be $m$ MAKESET, UNION and
		FINDSET operations, $n$ of which are MAKESET operations, where the
		total cost of all operations is $\theta(m \log n)$.


	\item Assume you are given a connected graph $G$.  Give an algorithm
		that returns a vertex $v$ in $G$, such that if $v$ is removed, $G$ is
		still connected.  Motivation: $G$ might represent a social network at
		a company and you want to choose some unlucky person to fire whose
		removal will not disconnect the company network.


	\item Professor Moe conjectures that for any graph $G$, the set of
		edges \{(u,v) : there exists a cut (S,V-S) such that (u,v) is a light
		edge crossing (S, V-S)\} always forms a minimum spanning tree.  Given
		a simple example of a connected graph that proves him wrong.


	\item Exercise 23.1-2 (``Professor Sabatier conjectures'')

	\item Exercise 23.1-3 (``Show that if an edge (u,v) is contained in
		some minimum spanning tree'')

	\item Exercise 22.2-6 / 22.2-7 (``There are two types of professional
		wrestlers'')

	\item Assume you are given a connected graph $G$.  Give an algorithm
		that returns a vertex $v$ in $G$, such that if $v$ is removed, $G$ is
		still connected.  Motivation: $G$ might represent a social network at
		a company and you want to choose some unlucky person to fire whose
		removal will not disconnect the company network.

\end{enumerate}









  \end{document}

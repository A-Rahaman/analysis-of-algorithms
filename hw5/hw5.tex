%%%%%latex preamble%%%%%
\documentclass[titlepage]{article}\usepackage[]{graphicx}\usepackage[]{color}
%% maxwidth is the original width if it is less than linewidth
%% otherwise use linewidth (to make sure the graphics do not exceed the margin)
\makeatletter
\def\maxwidth{ %
  \ifdim\Gin@nat@width>\linewidth
  \linewidth
  \else
  \Gin@nat@width
  \fi
}
\makeatother


\usepackage{listings}
\definecolor{mygreen}{rgb}{0,0.6,0}
\definecolor{mygray}{rgb}{0.5,0.5,0.5}
\definecolor{mymauve}{rgb}{0.58,0,0.82}
\lstset{ %
  backgroundcolor=\color{white},   % choose the background color; you must add \usepackage{color} or \usepackage{xcolor}
  basicstyle=\footnotesize,        % the size of the fonts that are used for the code
  breakatwhitespace=false,         % sets if automatic breaks should only happen at whitespace
  breaklines=true,                 % sets automatic line breaking
  captionpos=b,                    % sets the caption-position to bottom
  commentstyle=\color{mygreen},    % comment style
  deletekeywords={...},            % if you want to delete keywords from the given language
  escapeinside={\%*}{*)},          % if you want to add LaTeX within your code
  extendedchars=true,              % lets you use non-ASCII characters; for 8-bits encodings only, does not work with UTF-8
  frame=single,                    % adds a frame around the code
  keepspaces=true,                 % keeps spaces in text, useful for keeping indentation of code (possibly needs columns=flexible)
  keywordstyle=\color{blue},       % keyword style
  language=Python,                 % the language of the code
  morekeywords={*,...},            % if you want to add more keywords to the set
  numbers=left,                    % where to put the line-numbers; possible values are (none, left, right)
  numbersep=5pt,                   % how far the line-numbers are from the code
  numberstyle=\tiny\color{mygray}, % the style that is used for the line-numbers
  rulecolor=\color{black},         % if not set, the frame-color may be changed on line-breaks within not-black text (e.g. comments (green here))
  showspaces=false,                % show spaces everywhere adding particular underscores; it overrides 'showstringspaces'
  showstringspaces=false,          % underline spaces within strings only
  showtabs=false,                  % show tabs within strings adding particular underscores
  stepnumber=2,                    % the step between two line-numbers. If it's 1, each line will be numbered
  stringstyle=\color{mymauve},     % string literal style
  tabsize=2,                       % sets default tabsize to 2 spaces
  title=\lstname                   % show the filename of files included with \lstinputlisting; also try caption instead of title
}
\usepackage{alltt}
\usepackage[sc]{mathpazo}
\usepackage{amsmath, amsthm, amssymb}
\usepackage{graphicx}
\usepackage[T1]{fontenc}
\usepackage{geometry}
\geometry{verbose,tmargin=2.5cm,bmargin=2.5cm,lmargin=1.5cm,rmargin=1.5cm}
\setcounter{secnumdepth}{2}
\setcounter{tocdepth}{2}
\usepackage{url}
\usepackage[unicode=true,pdfusetitle,
  bookmarks=true,bookmarksnumbered=true,bookmarksopen=true,bookmarksopenlevel=2,
breaklinks=false,pdfborder={0 0 1},backref=false,colorlinks=false]
{hyperref}
\hypersetup{pdfstartview={XYZ null null 1}}
\usepackage{float}
\usepackage{bm}
\usepackage{tikz}
 %changes default sectioning commands -> 1,a, etc.
%\usepackage{breakurl}
\renewcommand{\thesubsection}{(\alph{subsection})}
\renewcommand{\thesubsubsection}{\roman{subsection}.}
\usepackage{lastpage}
\usepackage{fancyhdr}
\pagestyle{fancy}

%%% Header and Footer %%% 
\lhead{}
\chead{\leftmark}
\rhead{}
\lfoot{Aaron Gonzales; Algorithms}
\cfoot{Homework 5}
\rfoot{Page \thepage\ of \pageref{LastPage}}
\IfFileExists{upquote.sty}{\usepackage{upquote}}{}

\begin{document}

\title{Homework 5 \\ CS561 Fall 2014}
\author{Aaron Gonzales}
\maketitle


\section{Cartels}
  \begin{quote}
    \textbf{The Juarez Cartel is being investigated by the Federales. The $n$
    members of the cartel are organized in a pefect binary tree (recall that in a
  perfect binary tree, all leaves are at the same depth and each internal node
  has two children). Each day, the Federales select a node uniformly at random
  from the tree. They then arrest this node and all nodes below it - subordinates
  are arrested due to incriminating information obtained from the selected node.
  All arrested nodes are removed from the tree. \\ what is the expected number of
  days until all nodes are removed? Hint: for each node $v$ let $X_v$ be an IRV
  that is 1 if $v$ is arrested before any of its ancestors fand 0 otherwise. Use
  linearity of expectation. Your solution can use $\Theta$ notation. }
  \end{quote}

  Following the hint, if $X_v$ is our IRV, the probability of the IRV should be
  \[ \frac{1}{log \ell +1} \] where $\ell = height of tree$. We can then see that
  by linearity of expectation 
  \[ \sum_{i=0}^n \frac{1}{\ell +1} \] and instead of summing over the tree,
  we can notice that we can sum at each level of the tree and bound it from above
  by its height as was done in the class version of recursion trees/master
  method derivation. 

  \[ \sum_{i=0}^{\ell} \frac{2^i}{\ell +1} \]

  if we rewrite the sum as a recurrence relation:
  \[ A(i) = \sum_{i=0}^{\ell} \frac{2^i}{\ell +1} \]
  \[ A(i) = A(i-1) + \frac{2^i}{\ell +1} \]
  we can see that a trivial loose upper bound for this is $O(n)$ 
  and that a tighter bound is 
  \[ \Theta\left(\frac{n}{log n}\right) \] 
  days until everyone is arrested\footnote{trivial algebra omitted}. 



\section{Binary Search - Problem 17-2}
\begin{quote}
  \textbf{
	Binary search of a sorted array takes logarithmic search time, but the time
	to insert a new element is linear in the size of the array. We can improve
	the time for insertion by keeping several sorted arrays.  Specifically,
	suppose that we wish to support SEARCH and INSERT on a set of n elements.
	Let $k =  \lceil lg(n+1)\rceil $ and let the binary representation of n be
	$ \langle n_{ k-1 }, n_{k-2},\dots n_0 \rangle$.  We have k sorted arrays $
	A_0, A_1, \dots A_{k-1}$ where for $ i = 0, 1, \dots k-1$, the length of
	array $A_i$  is $2^i$ .  Each array is either full or empty, depending on
	whether $n_i = 1 $ or $n_i = 0$, respectively. The total number of elements
	held in all k arrays is therefore $\sum_{i = 0}^{k -1 } n_i 2^i = n $ .
	Although each individual array is sorted, elements in different arrays bear
	no particular relationship to each other.}
\end{quote}


\subsection{Describe how to perform the SEARCH operation for this data structure.
Analyze its worst-case running time.}
  \subsubsection{answer}
  Searching the whole data structure involves searching each of the
  sorted arrays in the structure. If an element isn't found in an array,
  we have to search another array, and so on, so in the worst case we'd
  have to perform binary search on all of the $k = O(lg n) $ arrays in
  the structure. Binary search takes $O(lg n)$ time and this sums to
  $O(lg^2 n)$ time. 

\subsection{Describe how to perform the INSERT operation. Analyze its
worst-case and amortized running times.  }
  \subsubsection{answer}
  Find the smallest array $A_i$ that is not empty and insert the new item
  into $A_i$ and move all elements from other arrays $A_0, \dots A_{i-1}$
  into $A_i$. In the worst case insertion, we have to moe all the current
  items in the structure (this can include the item being inserted). 

  For any sequence of $n$ insertions, $A_i$ will be changed from empty to
  full $\frac{n}{2^i}$ times. As such, $n$ operations takes:
  \[ \sum_{i=0}^{k} \frac{2^i *n }{2^i} = \sum_{i=0}^{k} n = O(n lg n)\] 
  and over $n$ operations, we get $O(lg n)$ amortized cost.


\subsection{Discuss how to implement DELETE.}
  \subsubsection{answer}
  Find $A_j$, the first full array; and
  $A_i$, the array to which $x$ belongs. Remove an element $p$ from
  $A_j$ and replace $x$ in $A_i$ with it and move  
  items from $A_j$ into $A_0, \dots, A_{j-1}$.

  In the worst case, we get a time of $O(n)$.
  which amounts to $O(\lg n)$ to find array $A_j$; $O(\lg^2 n)$ to 
  find $A_i$; $O(2^{\lg n}) = O(n)$ to insert $p$
  into $A_i$; and $O(n)$ to move
  $A_j$ to its underarrays. 

\section{Problem 22-4 (Reachability)}
\begin{quote}
\textbf{Let $G = (V,E) $ be a directed graph in which each vertex $ u \in V$
  is labeled with a unique integer $L(u) $ from the set $\{1,2,\dots,|V|\}$.
  For each vertex $u \in V$ let $R(u) = \{v \in V : u \rightsquigarrow v\}$ be
  the set of vertices that are reachable from $u$ Define $min(u)$ to be the
  vertex in $R(u)$ whose label is minimum, i.e., $min(u)$ is the vertex $v$
  such that $L(v) = min\{L(w) : w \in R(u)\}. $ give an $O(V + E)$ time
  algorithm that computes min(u) for all vertices $u \in V$.  }
\end{quote}
\subsubsection{answer}
N
If we define the transpose of a directed graph $G = G^T$ such that the graphs
edges are reversed (e.g., the edge $(a,b)$ becomes $(b,a)$). Feed the graph to
the following algorithm:
\begin{lstlisting}

findMinReachable(graph g):
	g_transpose = g.transpose()
	for (vertex.discovered == False) in g_transpose:
		r(vertex) = l(vertex)
		vertex.setDiscovered(True)
		for u in dfs(vertex)
			if u.discovered ==  False:
			r(u) = r(vertex)
			u.setDiscovered(True)
	  
\end{lstlisting}
The algorithm is backed by DFS and only searches vertices once, keeping its
native $O(V + E) $ running time. 


\section{union by rank}
\begin{quote}
  \textbf{Professor Curly conjectures that if we do union by rank,
\emph{without path compression}, the amortized cost of all operations
is $o(\log n)$.  Prove him wrong by showing that if we do union by
rank without path compression, there can be $m$ MAKESET, UNION and
FINDSET operations, $n$ of which are MAKESET operations, where the
total cost of all operations is $\theta(m \log n)$.}
\end{quote}

The running time of any findset operation in the tree is proportional to the
depth of the tree, which is the point of path compression. We know that the
number of total operations $m$ s greater than the total number of makeset
operations $n$. We know that $n$ operations at a cost of $log n$ reduces down
to an amortized cost of $log n$ and with up to $m$ total operations, we have an
overall amortized cost of $m log\, n$ for union without path compression.

\section{articulation points}
\begin{quote}
  \textbf{Assume you are given a connected graph $G$.  Give an algorithm
that returns a vertex $v$ in $G$, such that if $v$ is removed, $G$ is
still connected.  Motivation: $G$ might represent a social network at
a company and you want to choose some unlucky person to fire whose
removal will not disconnect the company network.}
\end{quote}
\subsubsection{Answer}
  This question is highly related to CLRS 22-2 and will follow most of it. 
  Some definitions to show:
  An articulation point is a point that when removed from a graph, disconnects
  the graph. (See page 621 of CLRS).
  $G_\pi = (V, E_\pi)$ is a DFS tree of $G$. 
  The root of $G_\pi$ is an articulation point iff it has at least two children
  in $G_\pi$. If this vertex is called $r$, if it has no children, then there are
  no other nodes from which it can disconnect the graph and it is not an
  articulation point. If $r$ has only one child, then every other point in
  $G_\pi$ is reachable through the child node and $r$ is not an articulation
  point. If the root has two children in $G_\pi$ then r is an articulation point.
  If the root r has two children, $c_1, c_2$, and $c_1$ was visited first by the
  DFS. If r is not an articulation point, then there must be another path between
  nodes in $C_1, C_2$ that doesn't include $r$. Exploring $C_1$ would have
  revealed this path and as such nodes of $C_2$ should be children of nodes in
  $C_1$. r is an articulation point because they are in disconnected subtrees of
  r and no path exists between them. 

  $v$ is a nonroot vertex $v \in G_\pi$. v is an articulation point of $G$ iff v
  has a child $c$ such that there is no back edge from $c$ or any decendent of
  $c$ to an ancestor of v. If $C_r$ is the connected component in $G - v $ that
  has the root of $G_\pi$. Let $s$ be a neighbor of $v$ that is not in $C_r$ and
  another compenent $C_s$ have $s$. All ancestors of $v$ are in $C_r$ and all
  decendents of s are in $C_s$. No edges between decendents of s and ancestors of
  v exist. If there $\exists s$ such that $s$ has no back edge from it or any of
  the set of its decendents to an ancestor of the node $v$, then there is only
  one path from the root of our tree goes to $s$ via $v$ and removing $v$ would
  disconnect the root of tree $G_\pi$ to $s$.

  Let a function $low(v)$ be the lowest number of the DFS tree of any vertex
  that is either in the dfs subtree with a root at $v$ (which includes v) or
  connected to a vertex in that subtree by a back edge. If a child $x$ of $v$
  has $low(v) \geq dfs_num(v)$ then v is an articulation point. 
\[
  v.low = min 
  \begin{cases}
	v.d, \\
        w.d: (u,w) \text{is a back edge for a decendent } u \text{of} v
  \end{cases}
\]


  An algorithm to find the articulation points of a graph could easily be
  modified to store non-articulation points as well. The algorithm is backed by
  DFS, which carries a quick runtime $O(V+E)$ and this particular algorithm
  keeps a seperate list of articulation points to query if someone wants to
  remove it from a graph. In a full implementation, a class Vertex could hold a
  field that denotes if it is an articulation point and could be found after
  the graph is initalized. 

  \begin{lstlisting}
  init: v.dfs_num = -1  for all v; DFS(v) for all unvisited v.
  articulation_points ={}
  DFS(v):
    v.dfs_num ++
    v.low = v.dfs_num // initialization
    for each edge (v,x)
      if x.dfs_num == -1 # x is undiscovered
        DFS(x)
        v.low = min(v.low, x.low)
        if (x.low >= v.dfs_num)
        articulation_points.add(v)
      else if x is not parent of v
        v.low = min(v.low, x.dfs_num)

    checkIfArticulationPoint(v):
      return articulation_points.contains(v)
  \end{lstlisting}

  



\section{Moe's graphs}
  \begin{quote}
    \textbf{Professor Moe conjectures that for any graph $G$, the set of
  edges \{(u,v) : there exists a cut (S,V-S) such that (u,v) is a light
  edge crossing (S, V-S)\} always forms a minimum spanning tree.  Given
  a simple example of a connected graph that proves him wrong.}
  \end{quote}
  \subsubsection{Answer}
  In the figure below, each edge has weight 1.
  \begin{verbatim}
       a
      / \
     b---c
  \end{verbatim}

\section{ Exercise 23.1-2 (``Professor Sabatier conjectures'')}
\begin{quote}
  \textbf{Professor Sabatier conjectures the following converse of Theorem
  23.1. Let $G = (V E)$  be a connected, undirected graph with a real-valued weight function w
defined on E. Let A be a subset of E that is included in some minimum spanning
tree for G, let $(S, V- S)$ be any cut of G that respects A, and let $u,v$ be a
safe edge for A crossing $(S, V-S)$. Then, $(u,v)$ is a light edge for the cut. Show
that the professor’s conjecture is incorrect by giving a counterexample.}
\end{quote}

A safe edge is defined to be any edge such that when we add it to a set of
edges $A$ we have a subset of a minimum spanning tree. There may be multiple
safe edges for a subset $S$ to vertices $V-S$ and not every safe edge must be
also be a light edge. 

\begin{verbatim}
    1     1
  a----b-----c
--|----|-----|-------C
  3    2     5
  |    |     |
  d	   e     f
\end{verbatim}

  only the edge $(b,e)$ is a light edge across the cut C but $(a,d)$ and
  $(c,f)$ are safe. 


\section{ Exercise 23.1-3 - MST} 
\begin{quote}
  \textbf{Show that if an edge $(u,v)$ is contained in some minimum spanning
  tree, then it is a light edge crossing some cut of the graph.}
\end{quote}
\subsubsection{answer}
If we have a graph $G$ with MST $T$ having edge $a = (u,v) \in T$, removing $a$
from $T$ would bifurcate $T$ into sets $R,S$. Then $a$ crosses the cut and if
it wasn't a light edge for the cut we would have taken the edge with lighter
weight isntead of $a$ to get a tree with lighter total weight than $T$. This
would show that $T$ was in fact not an MST and as such, edge $a$ must be a
light cut for the cut partitioning the graph into $R,S$. 




\section{Wrestlers Exercise 22.2-6 / 22.2-7 } 
\begin{quote}
  \textbf{There are two types of professional wrestlers: ``babyfaces'' and
  ``heels''. Between any pair of professional wrestlers, there may or may not
be a rivalry. Suppose we have $n$ professional wrestlers and we have a list of
r pairs of wrestlers for which there are rivalries. Give an $O(n + r)$ time
algorithm that determines whether it is possible to designate some of the
wrestlers as babyfaces and the remainder as heels such that each rvailry is
between a babyface and a heel. Ifi t is posible to perform such a designation,
your algorithm should produce it.}
\end{quote}

This is equivalent to finding if a graph is bipartite. We can say that a graph
$G$ is bipartite if it can be partitioned into two sets $X,Y$ such that every
edge has one end point in $X$ and the other end point in $Y$. 
If we let graph $G = (V,E)$ be the graph with each vertex representing a
wrestler and each edge representing a rivalry between wrestlers such that for
$n$ wrestlers and $r$ rivalries, we have 
$V = \{v_1, v_2, \dots v_n\} = n = \{n_1, n_2, \dots n\} \text{and} E \{e_1, e_2, \dots e_r\} = \{r_1, r_2, \dots r \}$.  

We can 'color' the unlabled graph as follows to find if it is bipartite. Pick a
vertex v and label it as ``babyface''. Start a BFS on v. 

For each vertex whose distance from the source label is even, label it as a
``babyface'' and for each vertex whose distance from the source label is odd, label
it a ``heel''. If any child has the same label assignment as it's parent, we
have a non-bipartite graph and can return an answer as such, and if the
algorithm labels everything without returning a failed value, we have a
bipartite coloring of the graph with babyfaces and heels.

\end{document}

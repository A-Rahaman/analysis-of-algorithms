%%%%%latex preamble%%%%%
\documentclass[titlepage]{article}\usepackage[]{graphicx}\usepackage[]{color}
%% maxwidth is the original width if it is less than linewidth
%% otherwise use linewidth (to make sure the graphics do not exceed the margin)
\makeatletter
\def\maxwidth{ %
  \ifdim\Gin@nat@width>\linewidth
  \linewidth
  \else
  \Gin@nat@width
  \fi
}
\makeatother


\usepackage{listings}
\definecolor{mygreen}{rgb}{0,0.6,0}
\definecolor{mygray}{rgb}{0.5,0.5,0.5}
\definecolor{mymauve}{rgb}{0.58,0,0.82}
\lstset{ %
  backgroundcolor=\color{white},   % choose the background color; you must add \usepackage{color} or \usepackage{xcolor}
  basicstyle=\footnotesize,        % the size of the fonts that are used for the code
  breakatwhitespace=false,         % sets if automatic breaks should only happen at whitespace
  breaklines=true,                 % sets automatic line breaking
  captionpos=b,                    % sets the caption-position to bottom
  commentstyle=\color{mygreen},    % comment style
  deletekeywords={...},            % if you want to delete keywords from the given language
  escapeinside={\%*}{*)},          % if you want to add LaTeX within your code
  extendedchars=true,              % lets you use non-ASCII characters; for 8-bits encodings only, does not work with UTF-8
  frame=single,                    % adds a frame around the code
  keepspaces=true,                 % keeps spaces in text, useful for keeping indentation of code (possibly needs columns=flexible)
  keywordstyle=\color{blue},       % keyword style
  language=Python,                 % the language of the code
  morekeywords={*,...},            % if you want to add more keywords to the set
  numbers=left,                    % where to put the line-numbers; possible values are (none, left, right)
  numbersep=5pt,                   % how far the line-numbers are from the code
  numberstyle=\tiny\color{mygray}, % the style that is used for the line-numbers
  rulecolor=\color{black},         % if not set, the frame-color may be changed on line-breaks within not-black text (e.g. comments (green here))
  showspaces=false,                % show spaces everywhere adding particular underscores; it overrides 'showstringspaces'
  showstringspaces=false,          % underline spaces within strings only
  showtabs=false,                  % show tabs within strings adding particular underscores
  stepnumber=2,                    % the step between two line-numbers. If it's 1, each line will be numbered
  stringstyle=\color{mymauve},     % string literal style
  tabsize=2,                       % sets default tabsize to 2 spaces
  title=\lstname                   % show the filename of files included with \lstinputlisting; also try caption instead of title
}
\usepackage{alltt}
\usepackage[sc]{mathpazo}
\usepackage{amsmath, amsthm, amssymb}
\usepackage{graphicx}
\usepackage[T1]{fontenc}
\usepackage{geometry}
\geometry{verbose,tmargin=2.5cm,bmargin=2.5cm,lmargin=1.5cm,rmargin=1.5cm}
\setcounter{secnumdepth}{2}
\setcounter{tocdepth}{2}
\usepackage{url}
\usepackage{hyperref}
\hypersetup{pdfborder = {0 0 0}}
\usepackage{float}
\usepackage{bm}
\usepackage{tikz}
 %changes default sectioning commands -> 1,a, etc.
%\usepackage{breakurl}
\renewcommand{\thesubsection}{(\alph{subsection})}
\renewcommand{\thesubsubsection}{\roman{subsection}.}
\usepackage{lastpage}
\usepackage{fancyhdr}
\pagestyle{fancy}
%\usepackage{tikz-qtree}

%%% Header and Footer %%% 
\lhead{}
\chead{\leftmark}
\rhead{}
\lfoot{Aaron Gonzales; Algorithms}
\cfoot{review problems}
\rfoot{Page \thepage\ of \pageref{LastPage}}
\IfFileExists{upquote.sty}{\usepackage{upquote}}{}

\begin{document}

\title{Homework 7, CS561, Fall 2014}
\author{Aaron Gonzales}
\maketitle


%%%% useful align for this
\section{CLRS 26-3, Algorithmic Consluting}
  \begin{quote}
    \textbf{Professor Gore wants to open up an algorithmic consulting company.
    He has identified $n$ important subareas of algorithms which he represents
    by the set $A = \{ A_1, A_2,\dots A_n\}$ In each subarea $A_k$ he can hire
    an expert in that area for $c_k$ dollars. The consulting company has lined
    up a set $J = (J_1, J-2 \dots J_m)$ of potential jobs. In order to perform
    job $J_i$ the company needs to have hried experts in a subset $R_i \in A$
    of subareas. Each expert can work on multiple jobs simulaneously. If the
    company chooses to accept job $J_i$, it must have hired eperts in all
  subares in $R_i$, and it will take in revenue of $p_I$ dollars.} 

    \textbf{Professor Gore's job is to determine which subareas to hire experts in and
    hwich jobs to accept in order to maimize the net revenue, which is the
    total income from jobs accepted minus the total cost of employing the
  experts.}

    \textbf{Consider the following flow network $G$. It contains a source vertex $s$,
    vertices $A_1, A_2,\dots A_n$, verices $J_1, J_2, \dots J_m$ and a sink
    vertex $t$. for $k = 1,2,\dots n,$ the flow network contains an edge
    $s<A_k$ with capacity $c(s,A_k) = c_k$,  and for $i = 1,2, \dots m$, the
    flow network contains an edge $(J_i, t)$ with capacity $c(J_i, t) = p_i$
    For $k = 1,2, \dots n$ and $i = 1,2, \dots m$ if $A_k \in R_i$ then $G$
  contains an edge $A_k, J_i)$ with capacity $c(A_k, J_i) = \infty$. }

  \end{quote}
 \subsection{ Show that if $J_i \in T$ for a finite-capacity cut $(S,T)$ of
 $G$, then $A_k \in T$ for each $A_k \in R_i$.}
 \vspace{8cm}

 \subsection{ Show how to determine the maxium net revenue from the capacity of
 a minmum cut of $G$ and the given $p_i$ values.}
 \vspace{8cm}


 \subsection{ Give an efficient algorithm to determine which jobs o accept and
   which experts to hire. Analyze the running time of your algorithm in terms
   of $m,n,$ and $r = \sum_{i = 1}^m | R_i|$.}



\end{document}

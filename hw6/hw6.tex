%%%%%latex preamble%%%%%
\documentclass[titlepage]{article}
\usepackage[]{graphicx}
\usepackage[]{color}
\usepackage{wrapfig}
%% maxwidth is the original width if it is less than linewidth
%% otherwise use linewidth (to make sure the graphics do not exceed the margin)
\makeatletter
\def\maxwidth{ %
  \ifdim\Gin@nat@width>\linewidth
  \linewidth
  \else
  \Gin@nat@width
  \fi
}
\makeatother


\usepackage{listings}
\definecolor{mygreen}{rgb}{0,0.6,0}
\definecolor{mygray}{rgb}{0.5,0.5,0.5}
\definecolor{mymauve}{rgb}{0.58,0,0.82}
\lstset{ %
  backgroundcolor=\color{white},   % choose the background color; you must add \usepackage{color} or \usepackage{xcolor}
  basicstyle=\footnotesize,        % the size of the fonts that are used for the code
  breakatwhitespace=false,         % sets if automatic breaks should only happen at whitespace
  breaklines=true,                 % sets automatic line breaking
  captionpos=b,                    % sets the caption-position to bottom
  commentstyle=\color{mygreen},    % comment style
  deletekeywords={...},            % if you want to delete keywords from the given language
  escapeinside={\%*}{*)},          % if you want to add LaTeX within your code
  extendedchars=true,              % lets you use non-ASCII characters; for 8-bits encodings only, does not work with UTF-8
  frame=single,                    % adds a frame around the code
  keepspaces=true,                 % keeps spaces in text, useful for keeping indentation of code (possibly needs columns=flexible)
  keywordstyle=\color{blue},       % keyword style
  language=Python,                 % the language of the code
  morekeywords={*,...},            % if you want to add more keywords to the set
  numbers=left,                    % where to put the line-numbers; possible values are (none, left, right)
  numbersep=5pt,                   % how far the line-numbers are from the code
  numberstyle=\tiny\color{mygray}, % the style that is used for the line-numbers
  rulecolor=\color{black},         % if not set, the frame-color may be changed on line-breaks within not-black text (e.g. comments (green here))
  showspaces=false,                % show spaces everywhere adding particular underscores; it overrides 'showstringspaces'
  showstringspaces=false,          % underline spaces within strings only
  showtabs=false,                  % show tabs within strings adding particular underscores
  stepnumber=2,                    % the step between two line-numbers. If it's 1, each line will be numbered
  stringstyle=\color{mymauve},     % string literal style
  tabsize=2,                       % sets default tabsize to 2 spaces
  title=\lstname                   % show the filename of files included with \lstinputlisting; also try caption instead of title
}
\usepackage{alltt}
\usepackage[sc]{mathpazo}
\usepackage{amsmath, amsthm, amssymb}
\usepackage{graphicx}
\usepackage[T1]{fontenc}
\usepackage{geometry}
\geometry{verbose,tmargin=2.5cm,bmargin=2.5cm,lmargin=1.5cm,rmargin=1.5cm}
\setcounter{secnumdepth}{2}
\setcounter{tocdepth}{2}
\usepackage{url}
\usepackage[unicode=true,pdfusetitle,
  bookmarks=true,bookmarksnumbered=true,bookmarksopen=true,bookmarksopenlevel=2,
breaklinks=false,pdfborder={0 0 1},backref=false,colorlinks=false,hidelinks]
{hyperref}
\hypersetup{pdfstartview={XYZ null null 1}}
\usepackage{float}
\usepackage{bm}
\usepackage{tikz}
 %changes default sectioning commands -> 1,a, etc.
%\usepackage{breakurl}
\renewcommand{\thesubsection}{(\alph{subsection})}
\renewcommand{\thesubsubsection}{\roman{subsection}.}
\usepackage{lastpage}
\usepackage{fancyhdr}
\pagestyle{fancy}
%\usepackage[active,tightpage]{preview}
\usepackage{tikz}
%\PreviewEnvironment{tikzpicture}
\theoremstyle{definition}
\newtheorem{definition}{Definition}[section]
%%% Header and Footer %%% 
\lhead{}
\chead{\leftmark}
\rhead{}
\lfoot{Aaron Gonzales; Algorithms}
\cfoot{Homework 6}
\rfoot{Page \thepage\ of \pageref{LastPage}}
\IfFileExists{upquote.sty}{\usepackage{upquote}}{}

\newtheorem{name}{Printed output}

\begin{document}

\title{Homework 6, CS561, Fall 2014}
\author{Aaron Gonzales}
\maketitle

\section{Naming Omicronians}
  \begin{quote}
    \textbf{Omicronians reproduce asexually with each baby producing up to two
    other babies. A birth process results in a binary tree, where babies are
    assigned their names in teh folloowing way. The baby at the root of the tree
    is assigned the name ``J''. Then any baby that is a left child of a node with
    the name $\sigma$ receives the name $\sigma L$ and any baby that is the right
    child of that node receives the name $\sigma R$. \\*
    Your goal is to take a binary tree specifying a birth process and return the
    number of R's in the names of all babies.}
  \end{quote}
  \begin{itemize}
    \item \textbf{\emph{Example:} in the following tree, the names are J, JL,
    JR, JLL, JLR, JRL, JRR, so the total number of R's is 5.  }
  \end{itemize}

  \subsection{For a node $v$ let $n(v)$ be the number of nodes in the subtree
    rooted at $v$ and assume this value is stored at each node. For a node $v$
    let $f(v)$ be the number of R's in the names of nodes below $v$. For
    simplicity, if $v$ is NULL, let $f(v) = 0$ and let $n(v) = 0$ Also, for a
    node $v$, let $l(v)$ (resp $r(v)$) be the left (resp. right ) child of $v$
    if it exists or NULL otherwise. Give a recurrence relation for $f(v)$.}
    \subsubsection{Answer:}
      \[ f(v) = f\left( l (v) \right) + f\left( r (v) \right) + n\left( r(v) \right) \]
      $f(v)$ is the sum of the number of r's in the names below the left subtree plus
      the number of r's in the right subtree plus the number of r's in the right
      subtree.


  \subsection{Your friend Bender claims that once you have the recurrence
    relation, you can just write a recursive algorithm for this problem in
    order to compute $f(r)$ where r is the root node. There's no need to bother
    with a dynamic program that stores the computed $f$ values at the nodes of
    the tree. Is Bender right? Explain in a couple of sentences.}
      \subsubsection{Answer:}
        \begin{wrapfigure}{l}{0.25\textwidth}
          \centering
            \includegraphics[scale=0.3]{bender}
        \end{wrapfigure}
        Yes, he is - there is no need to store all the previously calculated values in
        a table or other structure, you only need the current count. 

  \subsection{The Omicronians have evolved!} 
  \begin{quote}
    \textbf{Now they can reproduce both sexually
    and assexually: i.e., each node can have 1 or 2 parents. So a birth process
    is now a rooted directed graph without cycles. Now each node can have
    multiple names, one for each path from the root down to that node. A
    particular path determins a name in the same way as before: start with J at
    the root and add a L whenever a left edge is taken and a R whenever a right
    edge is taken. \\*
    Professor Farnsworth claims that the same recurrence relation from part a
    can be used for this problem. Bender claims that this problem can also be
    done recursively, without the dynamic programming approach of storing $f$
    values at intermediate nodes. Which if either of your friends are correct?
    Explain briefly.}
  \end{quote}
    \subsubsection{Answer: }
      \begin{wrapfigure}{r}{0.25\textwidth}
        \centering
          \includegraphics[scale=0.3]{farnsworth}
        \end{wrapfigure}
      Farnsworth is right. 



\section{Bins, Probability, and Expectation}
\begin{quote}
  \textbf{There are two bins. Bin 1 initially has 3 white balls and 1 red ball.
  Bin 2 has 4 white balls. In every round, a ball is selected uniformly at
  random from each bin and these two balls are swapped. \\*
  Let $p_k$ be the probability that the red ball is in bin 1 at the beginning of
  the $k$-th round.}
\end{quote}

\subsection{ Write a recurrence relation for $p_k$. }
  \subsubsection{Answer:}
    Note that $p_k$ is the probability that the red ball is in bin 1 at the beginning of
    the $k$-th round.

    We can write this approaching from two angles, one if the red ball is in the
    right bin and another if the red ball is in the left bin at the beginning of
    the $k$th round. 
    We know that if, at any round, the chance of the red ball moving from the right
    bin to the left if it is in the right bin is $\frac{1}{4}$ and the chance of it
    staying in the left bin if it is in the left bin is $\frac{3}{4}$. 

    A recurrence relation follows:
    \begin{align}
      p(k) & = \frac{1}{4}\left(1 - p(k-1)\right) + \frac{3}{4}\left(p\left(k-1\right)\right) \\*
      p(k) & = \frac{1}{4} - \frac{1}{4}p(k-1) + \frac{3}{4}p(k-1) \\*
      p(k) & = \frac{1}{2}p(k-1) + \frac{1}{4} 
    \end{align}


\subsection{Use the guess and check and proof by induction method to solve this
	recurrence, Don't forget to label BC, IH, and IS and clearly say where you
	are using the IH}.
  \subsubsection{Answer:}
    First, we solve the recurrence.

    \begin{align}
      p(k) & = 1/2 p(k -1) + 1/4 \\*
      p(k+1) & = 1/2 p(k)
      p(k) - 1/2 p(k) & =  0 
      Lp - 1/2 p & = 0
      \left( L - 1/2 \right) & = 0 \text{annihilates the sequences homogeneous
      part}
    \end{align}
    \begin{align}
      \left(L - 1\right) \text{annihilates the sequences non-homogeneous
      part, leaving us with }\\
      \left( L - 1/2 \right) \left(L - 1\right)
    \end{align}

    Solving for the constants gives us:
    \[ f(n) = c_1 * (1/2)^n + c_2 \]
    \begin{align}
      \text{base cases}: \\*
      p(1) = 1: & f(1) = c_1 (1/2)^1 + c_2 = 1  \\*
      p(2) = 3/4: & f(2) = c_1 (1/2)^2 + c_2 = 3/4
    \end{align}
    \begin{align}
      c_1 &= (1/2)^2 + (1 - 1/2 c_1) = 3/4 \\*
      c_1 &= \left( (1/2)^2 - (1/2)\right) \\*
          &= 3/4 - 1 \\*
          &= 1 \\*
    \end{align}
    \begin{align}
      c_2 &= 1 - 1/2 c_1 \\*
      c_2 &= 1 - 1/2 * 1 \\*
      c_2 = 1/2
    \end{align}
  We can say that 
  \[ p(k) = \left(\frac{1}{2}\right)^k + \frac{1}{2} \]
  and use it as a guess to prove via induction.
  \begin{proof}
      let $f(n) = \frac{1}{2} f(n-1) + \frac{1}{4}$ represent our recurrence. 
      suppose that :
    \[ f(n) = \left(\frac{1}{2}\right)^n + \frac{1}{2} \]
    We have two base cases:
    \[ n =1: f(n) = (1/2)^1 + 1/2 = 1 \]
    \[ n = 2: f(2) = (1/2)^2 + 1/2 = 3/4 \]

    Inductive Hypothesis:
    \[ \forall j < n, f(j) = (1/2)^j + 1/2 \]

    Inductive step:
    \begin{align}
      f(n) &= 1/2 f(n-1) + 1/4 \\*
      &= 1/2\left( (1/2)^{n-1} + 1/2 \right) + 1/4 \\*
      &= 1/2 \left( \frac{\left(1/2\right)^j}{1/2} + 1/2 \right) + 1/4 \\*
      &= (1/2)^j + 1/4 + 1/4 \\*
      &= (1/2)^j + 1/2 \qedhere
    \end{align}
  \end{proof}


  \subsection{Assume you are paid one dollar for every round in which the red
    ball is in bin 1 and there are $m$ rounds. What is the expected number of
    dollars you earn?}
    \subsubsection{Answer:}
      Let $X_k$ be an indicator random variable that denotes the event when the red
      ball is in bin 1. We define $X_k$ as 
      \begin{align}
        X_k = \begin{cases} 1 \text{ if red ball is in bin 1 at beginning of round }k \\*
                            0 \text{ otherwise }\\*
        \end{cases}
      \end{align}
      as such, we have:
      \begin{align*}
        E[X_k] &= Pr[X_k = 1] = \left(\frac{1}{2}\right)^k + 1/2  = \frac{1}{2^k} + 1/2 \\*
        E(x) &= E\left[ \sum_{k=1}^m X_k \right] \\*
        E(x) &= \sum_{k=1}^m \left[X_k \right] \text{ \emph{ By linearity of expectation}}\\*
        E(x) &= \sum_{k=1}^m \left( \left( \frac{1}{2^k}\right) + 1/2 \right) \\*
        E(x) &= \sum_{k=1}^m \left( \left( \frac{1}{2^m}\right) + 1/2 \right) \\*
        E(x) &= m \left( \left( \frac{1}{2^m}\right) + 1/2 \right) \\*
        E(x) &= \frac{m}{2^m} + \frac{m}{2}
      \end{align*}


\section{Tree Induction}
\begin{quote}
  \textbf{Prove via induction that any tree over n nodes has exactly $n-1$
  edges. }
\end{quote}

\subsubsection{Answer:}

\newtheorem{mydef}{Theorem}

\begin{name}
  Any tree over $n$ nodes has exactly $n-1$ edges.
\end{name}


\begin{proof}
By induction on $n$. \\*
Let $E$ be the number of edges in a tree.

Base case: when $n =1$, clearly $e = 0$ (there are no other nodes in the tree).

Inductive hypothesis: \\*
The Theorem is true for all trees with fewer than $ n$ vertices.

Let $T$ be a tree with $n$ nodes and $e$ be an edge from a node $u$ to a node
$v$. Only one path exists between $u,v$ and that is $e$. As such, if we delete $e$, the
tree $T$ is disconnected. $T - e$ represents the tree $T$ without the edge $e$
which can be represented by two subcomponents of $T$, $T_1, T_2$. Each
subcomponent is a tree, and let $n_1,n_2$ be the number of nodes in each
subcomponent such that $n_1 + n_2 = n$. 

Inductive step: \\*
By the inductive hypothesis, the number of edges in $T_1, T_2$ are $n_1-1,
n_2-1$
respectively and it follows that the tree $T$ has 
\[ E = n_1 -1 + n_2 -1 +1 = n_1 + n_2-1 = n-1 \]
  \qedhere
\end{proof}


\section{Claim 1 from SSSP}
  \begin{quote}
  \begin{itemize}
    \item \textbf{if $dist(v) \neq \infty $, then $dist(v)$ is the total weight
    of the predessor chain ending at $v$:
    \[ s \rightarrow \dots \rightarrow pred(pred(v)) \rightarrow pred(v) \rightarrow v \] }
    \item \textbf{This is easy to prove by induction on the number of edges in
    the path from $s$ to $v$.}
  \end{itemize}
  \end{quote}

  \subsubsection{Answer:}
    \begin{proof}
      By induction on $n$ where $n$ is the number of edges in the path from
      $s$ to $v$. 

      Base Case: 
      \[ s = v: dist(v) = 0 \]

      Inductive Hypothesis:
      \[ \forall j < n: dist(j) = dist(n) = \text{ total weight of pred ending at } v \]

      Inductive Step:
      $(s \rightarrow v)$ where $v = pred(v)$. By the inductive hypothesis,
      $dist(x) = pred(v)$ from $s \rightarrow v$, therefore $dist(v) = dist(x) +
      w(k,v)$
    \end{proof}




\section{Claim 2 from SSSP}
\begin{quote}
\begin{itemize}
  \item \textbf{ If the algorithm halts, then $dist(v) \leq w(s \leadsto v)$
	  for \emph{any} path $s \leadsto v$.}
  \item \textbf{This is easy to prove by induction on the number of edges in
	the path from $s$ to $v$.}
\end{itemize}
\end{quote}
\subsubsection{Answer:}

\begin{proof}
  The definition of relaxation follows:
  \begin{definition}
    We call an edge $(u, v)$ tense if $dist(u) + w(u, v) < dist(v)$ If $(u, v)$
    is tense, then the tentative shortest path from $(s \leadsto v )$ is
    incorrect since the path $(s \leadsto u)$ and then $(u, v)$ is shorter. Our
    generic algorithm repeatedly finds a tense edge in the graph and relaxes
    it. If there are no tense edges, our algorithm is finished and we have our
    desired shortest path tree.
  \end{definition}

  Let $k$ be the number of edges in $s \leadsto v$. 

\end{proof}


\section{Problem 23-4: Alternative MST}

\begin{quote}
  \textbf{ In this problem, we give pseudocode for three different algorithms.
	Each one takes a connected graph and a weight function as input and returns
	a set of edges $T$ .  For each algorithm, either prove that $T$ is a
	minimum spanning tree or prove that $T$ is not a minimum spanning tree.
	Also describe the most efficient implementation of each algorithm, whether or
	not it computes a minimum spanning tree.  }
\end{quote}

Before the problems, first a few definitions:

\theoremstyle{definition}
\begin{definition}{SpanningTree}
  A tree is a connected undirected graph with no cycles. It is a spanning tree
  of a graph $G$ if it spans $G$ (that is, it includes every vertex of $G$) and is a
  subgraph of $G$ (every edge in the tree belongs to $G$). A spanning tree of a
  connected graph $G$ can also be defined as a maximal set of edges of $G$ that
  contains no cycle, or as a minimal set of edges that connect all vertices.
  \footnote{verbatim from wikipedia, not from CLRS}
\end{definition}
\begin{definition}{Minimal}
  Minimum spanning tree. An edge-weighted graph is a graph where we associate
  weights or costs with each edge. A minimum spanning tree (MST) of an
  edge-weighted graph is a spanning tree whose weight (the sum of the weights
  of its edges) is no larger than the weight of any other spanning tree.
  \footnote{from princeton algs book}
\end{definition}

For all below answers, we have to show that for a an algorithm to return a
minimal spanning tree, it must meet both of the above criteria.

\subsection{}
\begin{lstlisting}
Maybe-MST-A(G, w)
	sort the edges into nonincreasing order of edge weights w
	T = E
	for each edge e, taken in nonincreasing order by weight
		if T-{e} is a connected graph
			T = T - {e}
	return T
\end{lstlisting}
\subsubsection{Answer: }
    \subsubsection{analysis}
      MaybeMSTA removes edges in nonincreasing order as long as the graph stays
      connected which results in a tree $T$ which is an MST. If we let $S$ be an
      MST $S \in G$  If we remove an edge from $g$, $e \in S$ or $e \notin S$. If
      $e \in S$, removing $e$ disconnects the tree $S$ into two subtrees. If a
      larger edge existed that connected the subtrees with lower weight than edge
      $e$, then the algorithm would have removed it before $e$ (nonincreasing order
      by weight). The graph is still connected, so anotehr edge with equal weight
      hasn't been discovered yet, safely allowing removal of $e$. 
      The algorithm returns a tree that is both a spanning tree and is minimal.
    \subsubsection{implementation}
      represent $T$ with an adjacency list and construct a priority queue from
      the egdes. Use union find with union-by-rank unions and path compression.
      We get a total running time of $O\left(E log(E)\right)$.


\subsection{}
\begin{lstlisting}
Maybe-MST-B(G,w)
  T = null
  for each edge e, taken in arbitrary order
	  if union(T, {e}) has no cycles
		  T = union(T, {e})
	  return T
\end{lstlisting}
  \subsubsection{Answer: }
  \subsubsection{analysis: }
    No, the algorithm does not return an MST. 
      
    \usetikzlibrary{arrows}
    \begin{figure}
      \begin{center}
      \begin{tikzpicture}[-,>=stealth',shorten >=1pt,auto,node distance=3cm,
        thick,main node/.style={circle,fill=blue!20,draw,font=\sffamily\Large\bfseries}]

        \node[main node] (1) {a};
        \node[main node] (2) [below of=1] {b};
        \node[main node] (3) [left  of=1] {c};

        \path[every node/.style={font=\sffamily\small}]
        (1) edge node [below right] {1} (2)
          %edge [bend right] node[left] {0.3} (2)
          %edge [loop above] node {0.1} (1)
        (2) edge node [below right] {1} (1)
            %edge [loop left] node {3} (3)
          %edge [bend right] node[left] {0.1} (3)
        (3) edge node [above] {2} (2)
            edge [bend left] node[below] {3} (1);
      \end{tikzpicture}
      \end{center}
    \caption{Figure blah}
    \label{fig:mst}
    \end{figure}

    The graph in Figure \ref{fig:mst} would have edges $\{a,b\}, \{b,c\}$ with
    weight 3. As cycles are taken in an arbitrary order it could try to add the
    edge $\{a,b\}$  to the previous picked sets $\{c,a\}, \{c,b\}$ which would add
    a cycle and result in a spanning tree, but that tree wouldn't be a minimal
    spanning tree. 

    \subsubsection{implementation:}
      Use a union-find data structure to hold $T$ in similar fashion to Kruskal's
      algorithm. We make $V$ calls to MakeSet, $2E$ find-set operations, and
      $V-1$ union operations, which gives us a $O(V log^* V + E)$ running time. 


\subsection{}
  \begin{lstlisting}
  Maybe-MST-C(G,w)
    T = null
    for each edge e, taken in arbitrary order
    T = union(T, {e})
    if T has a cycle c
      let e_prime be a maximum weight edge on c
      T = T - {e_prime}
    return T
  \end{lstlisting}

  \subsubsection{Answer:}
  \subsubsection{analysis:}
    $T$ will add every edge $e \in G$ and only edges $e$ that make cycles will
    be deleted. If a graph has an edge that results in a cycle, removing that
    edge removing that edge not disconnect the graph. Because $T$ has every
    edge added at some point and only delete edges that create cycles, $T$ is a
    spanning tree. 

    Adding any  edge within an MST creates a cycle, and $T$ is built in a way
    such that any edge that is added that creates a cycle is discarded, $T$ has
    the minimal number edges that comprise the set of edges that make a minimal
    spanning tree. The algorithm ensures that the maximal weighted edge within
    a cycle is removed, and following these steps, $T$ is both a spanning tree
    and minimal. 

  \subsubsection{implementation}
  Use an adjacency list for $T$ and when adding an edge to $T$, use DFS to
  check it for a cycle. If there are no cycles for the edge, we keep it in 
  $T$ and continue. If there is a cycle, we have to detect it's
  maximum-weighted edge and remove it from $T$. DFS takes $O(V+E)$ which in
  this algorithm is bounded by $O(V)$ as the edges in $T$ are never more than
  $V$ since cycles are removed. For each edge we have to perform DFS and this
  results in $O(E*V)$ running time. 



\section{Problem 24-2 Nesting Boxes}
  \begin{quote}
    \textbf{a d-dimensional box with dimensions $(x_1, x_2, \dots x_d)$
    \textbf{nests} within another box with dimensions $(y_1, y_2, \dots y_d)$
    if there exists a permutation $\pi$ on $\{1,2\dots,d\}$ such that 
    $ x_{\pi(1)} < y_1, x_{\pi(2)} < y_2, \dots x_{\pi(d)} < y_d$.  }
  \end{quote}

  \subsection{Argue that the nesting relation is transitive.}
    \subsubsection{Answer:}
      let $\mathcal R \subseteq S x S$ be a relation in $S$. 
      In order to show that a relation is transitive, we must show that  for

      \[ \left({x, y}\right) \in \mathcal R \land \left({y, z}\right) \in \mathcal R \implies \left({x, z}\right) \in \mathcal R \]

      \[ \left\{ {\left({x, y}\right), \left({y, z}\right)}\right\} \subseteq \mathcal R \implies \left({x, z}\right) \in \mathcal R \]
      In our problem, we must show that $x$ nests in $y$ and $y$ nests in some box
      $z$. If we find a relation $\pi$ for both of the relations above such that

      \[ x_{pi_i} < y_i \text{and } y_{pi_i} < z_j \]

      if our number of dimensions is the set $d$
      We have $ \{i,j\} \in d$, we can find for every $i$ a unique $j$ that satisfies
      $i = \pi_j$, resulting in 
      \[ x_{pi_i} < y_i =  y_{pi_i} < z_j \]
      which reduces to 
      \[ x_{pi_i} < z_j \]
      and the relation is transitive as we can find a unique $z_j$ for every
      $x_{\pi_i}$.



  \subsection{Describe an efficient method to determine weather or not one
              $d$-dimensional box nests inside another.}
  \subsubsection{Answer:}
    If we sort the $d$ dimension values within both $(x_1, x_2, \dots x_d)$ and
    $(y_1, y_2, \dots y_d)$ and then compare $x_i, y_i$ for every value of $i =
    1 \to d$ If $x_i <y_i$ for all $i$, this implies that $x$ nests within $y$.
    
  \subsection{Suppose that you are given a set of $n$ $d$-dimensional boxes
        $\{B_1, B_2,\dots B_n\}$. Give an efficient algorithm to find the longest
        sequence $\langle B_{i_1}, B_{i_2},\dots B_{i_k}\rangle$ of boxes such that
        $B_{i_j}$ nests within  $B_{i_{j+1}}$ for $j = 1,2,\dots,k-1$. Express the
        running time of your algorithm in terms of n and d.}
  \subsubsection{Answer:}
    As before, sort the dimension values $d$ within each box $(x_1, x_2, \dots x_d)$ and
    $(y_1, y_2, \dots y_d)$ Check if each possible pair of boxes $(B_i, B_j)$
    $B_i$ nests inside of $B_j$ or if the converse is true. Make a graph $G$
    with $n$ nodes where each node represents the one $B_i$, so that each
    $B_i, B_j$ pair where $B_i$ nests in $ B_j$ gets a directed edge from $B_i
    \to B_j$. Add a root vertex $s$ to the graph and give an edge from $r$ to
    every vertex in the graph. Add a 'destination' vertex $d$ to the graph
    that has an incoming edge from each vertex. Preform a topological sort of the
    graph with $s$ as the root vertex. Run a Bellman-Ford SSSP search on $s
    \rightarrow d$ . The most expensive part of the algorithm is building the
    graph, which involves $O(dn^2)$ operations, which is greater than the
    sorting time $O(nd log d)$ or search time $O(V + E)$. 



\section{Saia Trucking}
\begin{quote}
  \textbf{Saia Trucking is a very saftey concious (and algorithm loving)
  trucking company. They always try to find the safest route between any pair
of cities. They are thus faced with the following problem.}

\textbf{There is a graph $G = (V,E)$ where the vertices represent cities and the edges
represent roads. Each edge has a value associated with it that gives the
probability of safe transport on that edge i.e. the prbability that there will
be no accident when driving across that edge. The probability of safe transport
along any path in the gtraph is the product $\Pi$ of the probabilities of safe
transport on each edge in that path.}

\textbf{The Goal is to find a path from $s$ to $t$ that maximizes the probability of
safe transport. Describe an efficient algorithm to solve this problem.  }
\end{quote}
\subsubsection{Answer:}
We have $s$ as our start node and $t$ as our destination. let $e(u,v)$ be the
edge from vertex $u \rightarrow v$. We want to maximize the product of the
probabilities on the path such that the set of vertices $p = \langle v_0, v_1,
\dots, v_k \rangle$ where $v_0 = s, v_k = t$. We want

\[ p = max \prod_{i = 0}^{k} e(v_{i-1}, v_i) \]

If we log transform the edge weights, we can solve this problem easily. If we
define $w(e) = - log \, e(u,v)$ as the log transform of the edge weight, we can
change the problem to  a single-source shortest path problem as 

\[ p = min \sum_{i=1}^{k} w(v_{i-1}, v_i) \]

This is easily solved with Dijkstra's algorithm with the additional step of log
transforming the edges ($O(E)$). Using a Fibbonacci-heap-backed min-priority
queue we can do this is total $O(|E| + |V| log |V|)$ time.


\newpage
\section{25.2-1 - Floyd\textemdash Warshall}
\begin{quote}
  \textbf{Run the Floyd\textemdash Warshall algorithm on the weighted, directed
	graph of Figure \ref{fig:figure1}. Show the matrix $D^{(k)}$ that results for each
	iteration of the outer loop.}
\end{quote}

\usetikzlibrary{arrows}
\begin{figure}
  \begin{center}
	\begin{tikzpicture}[->,>=stealth',shorten >=1pt,auto,node distance=3cm,
	  thick,main node/.style={circle,fill=blue!20,draw,font=\sffamily\Large\bfseries}]

	  \node[main node] (1) {1};
	  \node[main node] (2) [right of=1] {2};
	  \node[main node] (3) [right of=2] {3};
	  \node[main node] (4) [below of=1] {4};
	  \node[main node] (5) [right of=4] {5};
	  \node[main node] (6) [right of=5] {6};

	  \path[every node/.style={font=\sffamily\small}]
		(1) edge node [below right] {-1} (5)
			%edge [bend right] node[left] {0.3} (2)
			%edge [loop above] node {0.1} (1)
		(2) edge node [above] {1} (1)
			edge node [below left] {2} (4)
			%edge [loop left] node {0.4} (2)
			%edge [bend right] node[left] {0.1} (3)
		(3) edge node [above] {2} (2)
			edge [bend left] node[right] {-8} (6)
		(4) edge node [left] {-4} (1)
			%edge [loop right] node {0.6} (4)
			edge [bend right] node[below] {3} (5)
		(5) edge node [right] {7} (2)
			%edge [loop right] node {0.6} (4)
		(6) edge node [below] {5} (2)
			edge [bend left] node {10} (3);
	\end{tikzpicture}
  \end{center}
\caption{Figure 25.2 from CLRS}
\label{fig:figure1}
\end{figure}
\subsubsection{Answer:}
\vspace{6cm}

\end{document}

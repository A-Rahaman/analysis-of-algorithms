%%%%%latex preamble%%%%%
\documentclass[titlepage]{article}\usepackage[]{graphicx}\usepackage[]{color}
%% maxwidth is the original width if it is less than linewidth
%% otherwise use linewidth (to make sure the graphics do not exceed the margin)
\makeatletter
\def\maxwidth{ %
  \ifdim\Gin@nat@width>\linewidth
  \linewidth
  \else
  \Gin@nat@width
  \fi
}
\makeatother


\usepackage{listings}
\definecolor{mygreen}{rgb}{0,0.6,0}
\definecolor{mygray}{rgb}{0.5,0.5,0.5}
\definecolor{mymauve}{rgb}{0.58,0,0.82}
\lstset{ %
  backgroundcolor=\color{white},   % choose the background color; you must add \usepackage{color} or \usepackage{xcolor}
  basicstyle=\footnotesize,        % the size of the fonts that are used for the code
  breakatwhitespace=false,         % sets if automatic breaks should only happen at whitespace
  breaklines=true,                 % sets automatic line breaking
  captionpos=b,                    % sets the caption-position to bottom
  commentstyle=\color{mygreen},    % comment style
  deletekeywords={...},            % if you want to delete keywords from the given language
  escapeinside={\%*}{*)},          % if you want to add LaTeX within your code
  extendedchars=true,              % lets you use non-ASCII characters; for 8-bits encodings only, does not work with UTF-8
  frame=single,                    % adds a frame around the code
  keepspaces=true,                 % keeps spaces in text, useful for keeping indentation of code (possibly needs columns=flexible)
  keywordstyle=\color{blue},       % keyword style
  language=Python,                 % the language of the code
  morekeywords={*,...},            % if you want to add more keywords to the set
  numbers=left,                    % where to put the line-numbers; possible values are (none, left, right)
  numbersep=5pt,                   % how far the line-numbers are from the code
  numberstyle=\tiny\color{mygray}, % the style that is used for the line-numbers
  rulecolor=\color{black},         % if not set, the frame-color may be changed on line-breaks within not-black text (e.g. comments (green here))
  showspaces=false,                % show spaces everywhere adding particular underscores; it overrides 'showstringspaces'
  showstringspaces=false,          % underline spaces within strings only
  showtabs=false,                  % show tabs within strings adding particular underscores
  stepnumber=2,                    % the step between two line-numbers. If it's 1, each line will be numbered
  stringstyle=\color{mymauve},     % string literal style
  tabsize=2,                       % sets default tabsize to 2 spaces
  title=\lstname                   % show the filename of files included with \lstinputlisting; also try caption instead of title
}
\usepackage{alltt}
\usepackage[sc]{mathpazo}
\usepackage{amsmath, amsthm, amssymb}
\usepackage{graphicx}
\usepackage[T1]{fontenc}
\usepackage{geometry}
\geometry{verbose,tmargin=2.5cm,bmargin=2.5cm,lmargin=1.5cm,rmargin=1.5cm}
\setcounter{secnumdepth}{2}
\setcounter{tocdepth}{2}
\usepackage{url}
\usepackage[unicode=true,pdfusetitle,
  bookmarks=true,bookmarksnumbered=true,bookmarksopen=true,bookmarksopenlevel=2,
breaklinks=false,pdfborder={0 0 1},backref=false,colorlinks=false]
{hyperref}
\hypersetup{pdfstartview={XYZ null null 1}}
\usepackage{float}
\usepackage{bm}
\usepackage{tikz}
 %changes default sectioning commands -> 1,a, etc.
%\usepackage{breakurl}
\renewcommand{\thesubsection}{(\alph{subsection})}
\renewcommand{\thesubsubsection}{\roman{subsection}.}
\usepackage{lastpage}
\usepackage{fancyhdr}
\pagestyle{fancy}

%%% Header and Footer %%% 
\lhead{}
\chead{\leftmark}
\rhead{}
\lfoot{Aaron Gonzales; Algorithms}
\cfoot{Homework 2}
\rfoot{Page \thepage\ of \pageref{LastPage}}
\IfFileExists{upquote.sty}{\usepackage{upquote}}{}

\begin{document}
\title{Homework 2, CS561, Fall 2014}
\author{Aaron Gonzales}
\maketitle

\section{ Skip lists }
\textbf{
\begin{quote}
	In this problem you will use Chernoff bounds to show that for most of the levels
	of a skip list, the size of the level is very tightly bounded around its
	xpectation. \\
	Chernoff Bounds: Assume you have n independent, indicator random variables
	$X_1$,
	$X_2$, \dots, $X_n$ and let $X = \sum_{i=0}^n X_i $ and $\mu = E(X)$. Then Chernoff bounds tell us
	that for any $0 \leq \delta \leq 1$: \\
	\[ Pr(X \leq (1 - \delta)\mu)  \text{ or }  X \, \geq (1 + \delta)\mu) \leq 2e^{−\mu \, \frac{\delta^2}{4}} \]
	Use Chernoff and Union bounds to show that with probability at least
	$1 − \frac{1}{n}$, for
	all $ 0 \leq j \leq log n - log log\, n - 5$ , List j in a skip list contains
	between $\frac{n}{2^{j+1}}$ 
	and $ \frac{3n}{2^{j+1}}$  nodes. Let List 0 be the bottom list, List 1 be the next higher up,
	etc. You may assume that n is sufficiently large, e.g. n is larger than some
	constant$n_0$. \\
	Note: Chernoff bounds are more powerful than bounds on Binomial distributions
	since X need not be binomially distributed. The only requirement is that
	the $X_i$ be independent. Hint: Remember that $e^{−x} \leq 2^{−x}$.
\end{quote}
}




\section{pebbles and graphs}

\begin{quote}
	\textbf{You toss m pebbles onto the n nodes of a k-regular undirected graph (recall
		that a graph is k-regular if every node has degree k). Each pebble lands on
		a node selected uniformly at random. A pair of pebbles is said to ``collide''
		if they fall on the same node or on two nodes that are neighbors. What is
		the expected number of pairs of pebbles that collide? About how large must
	m be before you would expect at least 1 pair of pebbles to collide?}
\end{quote}

We can use the Birthday Paradox to help us out here. Knowing that the BP says:
\[ E(X) = \frac{m(m-1)}{2n} \]
where $m = \text{people or pebbles}$ and $n = \text{number of days or nodes}$
and $E(X)$ is the expected number of pairs of people or pebbles that have the
same birthday or node. In our problem let us state:\\

Let $X{i_j}$ be an indicator variable saying 

\[
	X_{i,j} = I[X]
	\begin{cases}
		1 & \text{if pebbles collide (same or adjacent node)} \\
		0 & \text{if not } 
	\end{cases}
\]
and for all pairs of pebbles $i,j$ the probability of a collision is
$\frac{K+1}{n}$.


\[ \sum_{i=1}^{m} X_{i,j} \] 

\[ {m \choose 2} \frac{k+1}{n} = \frac{m(m-1)(k+1)}{2(n)} \]



\section{Recurrence}

\begin{quote}
	Consider the Recurrence $f(n) = 3f(\frac{n}{2}) + \sqrt{n}$
\end{quote}

\subsection{Use the Master method to solve this recurrence}
The generic form of the Master equation is:
\[ T(n) = aT\left(\frac{n}{b} + f(n)\right) \text{ where } a\geq 1, b> 1 \]
where:
$T(n)$ has the following asymptotic bounds:
\begin{enumerate}
	\item if $f(n) = O\left(n^{log_b a-\epsilon}\right)$ for some constast
		$\epsilon > 0$, then $T(n) = \Theta(n^{log_b a})$
	\item if $f(n) = \Theta(n^{log_b a}), \text{then} T(n) =
		\Theta\left(n^{log_b a} lg\,n\right)$. 
	\item if $f(n) = \Omega\left(n^{log_b a + \epsilon}\right)$ for some
		constant $c < 1$ and all sufficiently large $n$, then $T(n) =
		\Theta(f(n))$. 
\end{enumerate}

in our example, we have $a = 3$, $b = 2$, and $f(n) = \sqrt{n}$. 

\[ n^{log_b a} = n^{log_2 3} = O(n^{1.58}) \]

and $\sqrt{n} = n^{1/2} = O(n^{1.58})$ ,so $T(n) = \Theta(n^{1.58})$.

\subsection{Now use annihilators (and a transformation) to solve the recurrence. Show
your work. (This is perhaps stating the obvious, but please note that your two
bounds should match)}
Our recurrence is: $f(n) = 3f(\frac{n}{2}) + \sqrt{n}$. 

%\[ T(n) = 3T(\frac{n}{2}) + n^{\frac{1}{2}} \] 

and needs a transformation. Let's  let $n = 2^i$ and rewrite $T(n)$ as:

\[ T(2^i) = 3T(\frac{2^i}{2}) + 2^{i/2} \]
\[ T(2^i) = 3T(2^{i-1}) + 2^{i/2} \]
let $t(2^i)$ = $t(i)$
\[ t(i) = 3T(i-1) + \frac{i}{2} \]

Using annhilators, we can eliminate the sequence $ t = \langle t_i \rangle $. 

\[ (L-3)   \]

From the lookup table $(L-3) (L-1)^2$ will annihilate relations with the form
of $c_0*3^i + c_1i + c_2$

\section{More recurrence}

\begin{quote}
consider the following function:
\begin{lstlisting}
	int f (int n){
		if (n == 0) return 2;
		else if (n&=& 1) return 5;
		else{
			int val = 2*f (n-1);
			val = val - f (n-2);
			return val;
		}
	}
\end{lstlisting}
\end{quote}

\subsection{Write a recurrence relation for the value returned by f. Solve the
recurrence exactly. (Don't forget to check it.)}
We can state this algorithm as follows:
\[ f(n) = 2f(n-1) - f(n-2) \]
with a base case of $f(0) = 2$. For the following values of n, we get:

\begin{align}
	f(0) &= 2 \\
	f(1) &= 5 \\
	f(2) &= 8 \\
	f(3) &= 11 \\
	f(4) &= 14
	\label{eqn:something}
\end{align}

Formally stating our recurrence as 
for all values of $n > 0$,  i assume it returns $3n+2$. 
\begin{proof} By induction: for 
\[ f(n) = 2f(n-1) - f(n-2) , f(0) = 2,\,\, f(n) = 3n +2 \]

Base case: $f(0) = 2$.  
Inductive Hypothesis: $\forall j < n, f(j) = 3j + 2 $ \\
Inductive step: 

\begin{align*}
	f(n) &= 2f(n-1) - f(n-2) \\
	&= 2(3n + 2 -1) - 3n+2-2 \text{ by I.H.} \\
	&= 6n + 2 - 3n \\
	&= 3n + 2 \qedhere
\end{align*}
\end{proof}


\subsection{(b) Write a recurrence relation for the running time of f. Get a tight
upperbound (i.e. big-O) on the solution to this recurrence.}

\[ f(n) = 2f(n-1) - f(n-2) + \Theta(1) \]

We can see that this is similar to Fibonacci and use a few annihilators to
solve this.For the homogenous parts:
\[ T = \langle T_0, T_1, T_2, T_3, \ldots \rangle \]
\[ LT = \langle T_1, T_2, T_3, T_4, \ldots \rangle \]
\[ L^2T = \langle T_2, T_3, T_4, T_5, \ldots \rangle \]

\[ (L^2 - L - 1) \] 
In class, we learned that this factors to \[ (L	 - \phi)(L - \hat{\phi} ) \]

And our sequence must have the form 

\[ O(\phi^n + \hat{\phi}^n) \]
and reduces to 
\[ O(\phi^n) \]
\section{Silly-Sort }
\begin{lstlisting}
	Silly-Sort(A,i,j):
		if A[i] > A[j]:
			then exchange A[i] and A[j]
		if i+1 >= j:
			then return
		k = floor(j-i+1)/3)
		Silly-Sort(A,i,j-k)
		Silly-Sort(A,i+k,j)
		Silly-Sort(A,i,j-k)
	
\end{lstlisting}

\subsection{ Argue (by induction) that if n is the
length of A, then Silly- Sort(A,1,n)
correctly sorts the input array
$A[1\dots n]$.}

We can see that Silly-Sort is sorting $2/3$ of the list on each recursive call.
If we define a base case of a list where $n=3$, and the list is $\{ 2,3,1 \} $ the array will be sorted as
such:

\[ \{ 2, 3, 1 \} \] 
  \[ \{ 1, 3, 2 \} \]
  \[ \{ 1, 2, 3 \} \]
Inductive Hypothesis: Silly sort sorts any array length $j<n$, and knowing that
Silly-Sort will always eventually divide the list into length 3, by the base
case we can see that if the sublists of length three are sorted, the full list
would be sorted as well. 


\subsection{ Give a recurrence relation for the worst-case run time of Silly-Sort and a
tight bound on the worst-case run time}

We can define a recurrence relation for the algorithm as 
\[ T(n) = 3T(\frac{2n}{3}) + \Theta(1) \]
%and solve it by the master method:

%$a = 3, \, b = 2/3, \, c=4$ which gives us case 

Using an annihilator, we can elimiatesolve the recurrance by:

let $n = 2\times3^i$
\[ T(2*3^i) = 3T(2*\frac{3^i}{3}) + 4 \]
\[ T(2*3^i = 6T(3^{i-1}) + 4 \]
let $t(i) = 3^i$
\[ t(i) = 6t(i-1) + 4 \]
\[ (L-6)(L-1) \]
\[ c_16^i + c_21^i \]

\[ n = 2\times3^i \]
\[ lg n = lg2 + i lg3 \] 
back substituting we get:
\[ lg n - lg 2 = i lg3 \]
\[ \frac{lg n - lg 2}{lg 3} = i \]

\[ 6^{3^i} + 1^{3^i} \]
and reducing we get:

\[ 3^i + 3^i \]
\[ 3^i \]

  
\subsection{ Compare this worst-case runtime with that of insertion sort,merge
sort, heapsort and quicksort.}

\begin{listing}
\item Insertion sort: $O(n^2)$
\item Merge sort: $O(nlgn)$
\item heapsort: $O(n lg n)$
\item quicksort: $O(n^2)$ (Worst case, average is $O(n lg n)$
\item Silly Sort: 
\end{listing}


\section{Primes and Probability}

\begin{quote}
	\textbf{
In this problem, you will use the following facts. 1) any integer can be
uniquely factored into primes; 2) the number of primes less than any number m
is $\Theta(m/ log m)$ (this is the prime number theorem). \\
We will also make use of the following notation for integers x and y: \\
1) $x|y$ means that x ``divides'' y, which means that there is no remainder
when you divide y by x. \\ 
and 2) $ x = y(modp)$ means that x and y have the same remainder when
divided by p, or in other words, $p|x−y$} 
\end{quote}

\subsection{Show that for any integer x, x factors into at most log x primes. Hint: 2 is
the smallest prime.}
If we factor a number x using a binary tree, we can see that the greatest
number of prime factors will be less than or equal to the height of a binary
tree, which is $lg n$ in height. An example:

\begin{verbatim}
      8             15             16
    4   2          3  5         8      2
   2 2                       4     2
                           2  2
\end{verbatim}

\subsection{ Let x be some positive integer and let p be a prime chosen uniformly at
random from all primes less than or equal to m. Use the prime number theorem to
show that the probability that $p|x$ is $O\left( log x)(log m)/m\right)$. }

Let $X_i$ be an indicator variable saying 

\[
	X_i = I[X]
  \begin{cases}
	1 & \text{if prime factor } i \text{ of } x = p \\
	0 & \text{if not } 
  \end{cases}
\]

The probability of $X_{i,p}$ $\leq \frac{\frac{1}{m}}{\Theta log m} = \Theta
\frac{log m}{m} $


\subsection{ Now let x and y both be positive integers less than n and let p be a prime
chosen uniformly at random from all primes less than or equal to m. Using the
previous result, show that the probability that $ x ≡ y (mod p)$  is $ O\left( log
n)(log m)/m)\right)$ . }

\subsection{ If $ m = log^2 n$  in the previous problem, then what is the probability that
$x ≡ y (mod p)$. Hint: If you're on the right track, you should be able to show
that this probability is ``small'', i.e. it goes to 0 as n gets large. }

\subsection{ Finally, show how to apply this result to the following problem. Alice and
Bob both have databases x and y where x and y have value no more than n, for n
a very large number (think terabytes). They want to check to see if their
databases are consistent (i.e. they want to check if they are the same) but
Alice does not want to have to send her entire database to Bob. What is an
algorithm Alice and Bob can use to check consistency with reasonably good
probability by sending a lot fewer bits? How many bits does Alice need to send
to Bob as a function of n, and what is the probability of failure, where
failure means that this algorithm says the databases are the same but in fact
they are different? }


\section{Bad Santa }
\begin{quote}
	\textbf{
		A child is presented with n boxes, one after another. Upon receiving
		a box, the child must decide whether or not to open it. If the child does not
		open a box, he is never allowed to revisit it. Half the
		boxes have presents in them, but the decision about which boxes have presents
		is made by an omniscient and malicious Santa who wants the child to open as
		many empty boxes as possible before finding a present. \\ 
		Devise and analyze a randomized algorithm for the child which minimizes the
		expected number of boxes that need to be opened before the child finds the
		first present. Assume Santa knows your algorithm, but can not predict the
		random choices made by your algorithm. \\ 
		Hint: Birthday paradox. \\
		Note: This problem has applications to wireless networks: basically boxes are
		time-steps, Santa is a jamming adversary, and opening a box means spending
	energy to listen in a time-step.}
\end{quote}

Our problem has the following setup:

\begin{itemize}
	\item $n$ boxes
	\item $\frac{n}{2}$ boxes have presents
	\item Birthday paradox tells us that we need $\sqrt{n}$ pebbles to fit
		$n$ bins. 
\end{itemize}

If we naively pick boxes in order, santa will obviously put the boxes in the
last half of our 'box array'. if this is true, we will have to pick
$\frac{n}{2} + 1 = O(n)$ boxes to get a gift. 

Randomization helps us greatly here as Santa cannot account for our random
choices. If we divide the array in half we can find a lower bound than $O(n)$
for our number of opened boxes. 
As the birthday paradox was discussed in class, we can expect that given n bins
and p pebbles tossed randomly into bins, $\sqrt(n)$ tosses gives us an
expectation of 1 ``collision'' or two pebbles going into the same bin. If we
think of our pick of a box as a pebble and santa's placement of a box as a
pebble, 




\end{document}

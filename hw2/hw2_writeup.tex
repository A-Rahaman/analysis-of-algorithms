%%%%%latex preamble%%%%%
\documentclass[titlepage]{article}\usepackage[]{graphicx}\usepackage[]{color}
%% maxwidth is the original width if it is less than linewidth
%% otherwise use linewidth (to make sure the graphics do not exceed the margin)
\makeatletter
\def\maxwidth{ %
  \ifdim\Gin@nat@width>\linewidth
  \linewidth
  \else
  \Gin@nat@width
  \fi
}
\makeatother


\usepackage{listings}
\definecolor{mygreen}{rgb}{0,0.6,0}
\definecolor{mygray}{rgb}{0.5,0.5,0.5}
\definecolor{mymauve}{rgb}{0.58,0,0.82}
\lstset{ %
  backgroundcolor=\color{white},   % choose the background color; you must add \usepackage{color} or \usepackage{xcolor}
  basicstyle=\footnotesize,        % the size of the fonts that are used for the code
  breakatwhitespace=false,         % sets if automatic breaks should only happen at whitespace
  breaklines=true,                 % sets automatic line breaking
  captionpos=b,                    % sets the caption-position to bottom
  commentstyle=\color{mygreen},    % comment style
  deletekeywords={...},            % if you want to delete keywords from the given language
  escapeinside={\%*}{*)},          % if you want to add LaTeX within your code
  extendedchars=true,              % lets you use non-ASCII characters; for 8-bits encodings only, does not work with UTF-8
  frame=single,                    % adds a frame around the code
  keepspaces=true,                 % keeps spaces in text, useful for keeping indentation of code (possibly needs columns=flexible)
  keywordstyle=\color{blue},       % keyword style
  language=Python,                 % the language of the code
  morekeywords={*,...},            % if you want to add more keywords to the set
  numbers=left,                    % where to put the line-numbers; possible values are (none, left, right)
  numbersep=5pt,                   % how far the line-numbers are from the code
  numberstyle=\tiny\color{mygray}, % the style that is used for the line-numbers
  rulecolor=\color{black},         % if not set, the frame-color may be changed on line-breaks within not-black text (e.g. comments (green here))
  showspaces=false,                % show spaces everywhere adding particular underscores; it overrides 'showstringspaces'
  showstringspaces=false,          % underline spaces within strings only
  showtabs=false,                  % show tabs within strings adding particular underscores
  stepnumber=2,                    % the step between two line-numbers. If it's 1, each line will be numbered
  stringstyle=\color{mymauve},     % string literal style
  tabsize=2,                       % sets default tabsize to 2 spaces
  title=\lstname                   % show the filename of files included with \lstinputlisting; also try caption instead of title
}
\usepackage{alltt}
\usepackage[sc]{mathpazo}
\usepackage{amsmath, amsthm, amssymb}
\usepackage{graphicx}
\usepackage[T1]{fontenc}
\usepackage{geometry}
\geometry{verbose,tmargin=2.5cm,bmargin=2.5cm,lmargin=1.5cm,rmargin=1.5cm}
\setcounter{secnumdepth}{2}
\setcounter{tocdepth}{2}
\usepackage{url}
\usepackage[unicode=true,pdfusetitle,
  bookmarks=true,bookmarksnumbered=true,bookmarksopen=true,bookmarksopenlevel=2,
breaklinks=false,pdfborder={0 0 1},backref=false,colorlinks=false]
{hyperref}
\hypersetup{pdfstartview={XYZ null null 1}}
\usepackage{float}
\usepackage{bm}
\usepackage{tikz}
 %changes default sectioning commands -> 1,a, etc.
%\usepackage{breakurl}
\renewcommand{\thesubsection}{(\alph{subsection})}
\renewcommand{\thesubsubsection}{\roman{subsection}.}
\usepackage{lastpage}
\usepackage{fancyhdr}
\pagestyle{fancy}

%%% Header and Footer %%% 
\lhead{}
\chead{\leftmark}
\rhead{}
\lfoot{Aaron Gonzales; Algorithms}
\cfoot{Homework 2}
\rfoot{Page \thepage\ of \pageref{LastPage}}
\IfFileExists{upquote.sty}{\usepackage{upquote}}{}

\begin{document}

\title{Homework 2, CS561, Fall 2014}
\author{Aaron Gonzales}
\maketitle


%%%% useful align for this
\section{Template math}
\begin{align*}
	Total Matches &={n \choose 3} * \frac{1}{25} \\
					&= {27 \choose 3}   * \frac{1}{25} \\
					&=   \frac{27!}{3!(27-3)!} * \frac{1}{25} \\
					&= \frac{27 * 26 * 25}{6} * \frac{1}{25}  \\
					&= 117
\end{align*}


We have $n$ debutants and $n$ porsches. 
$\frac{1}{n}$ is our probability of a debutant getting into her own porche.
Let $X_i$ be an indicator variable saying 

\[
	X_i = I[X]
	\begin{cases}
		1 & \text{if debutante gets in her own car} \\
		0 & \text{if not} 
	\end{cases}
\]

The number of debutants who return to their own can be expressed as:
\[ X_i = \sum_{i=1}^n X_i \]

and the expected number of debutants can be expressed as:

\[ E [X_i] = E \left[ \sum_{i=1}^n X_i \right] \]

By linearity of expectation we can state it as such:

\begin{align*}
	E [X_i] &= E \left[ \sum_{i=1}^n X_i \right] \\
	E [X_i] &=  \sum_{i=1}^n E[X_i] \\ 
	&=  \sum_{i=1}^n \frac{1}{n} \\ 
	&=  \frac{1}{n} \sum_{i=1}^n 1 \\
	&= \frac{n}{n} \\
	& = 1
\end{align*}
We only expect one drunken debutant to get in her own porche. (Note - This presumes
that all of the debutants made it out of the party and no one passed out
inside.)













\section{ Skip lists }
\begin{quote}
	In this problem you will use Chernoff bounds to show that for most of the levels
	of a skip list, the size of the level is very tightly bounded around its
	xpectation. \\
	Chernoff Bounds: Assume you have n independent, indicator random variables
	$X_1$,
	$X_2$, \dots, $X_n$ and let $X = \sum_{i=0}^n X_i $ and $\mu = E(X)$. Then Chernoff bounds tell us
	that for any $0 \leq \delta \leq 1$: \\
	$ Pr(X \leq (1 − \delta)\mu) $ or$  X \geq (1 + \delta)\mu) \leq
	2e^{−\mu\delta^2 / 4} $ \\
	Use Chernoff and Union bounds to show that with probability at least
	$1−1/n$, for
	all $ 0 \leq j \leq log n − log log n-5$ , List j in a skip list contains
	between $n/2j+1$ 
	and $ 3n/2j+1$  nodes. Let List 0 be the bottom list, List 1 be the next higher up,
	etc. You may assume that n is sufficiently large, e.g. n is larger than some
	constant$n_0$. \\
	Note: Chernoff bounds are more powerful than bounds on Binomial distributions
	since X need not be binomially distributed. The only requirement is that the Xi
	be independent. Hint: Remember that $e^{−x} \leq 2^{−x}$.
\end{quote}





\section{pebbles and graphs}

\begin{quote}
	You toss m pebbles onto the n nodes of a k-regular undirected graph (recall
	that a graph is k-regular if every node has degree k). Each pebble lands on
	a node selected uniformly at random. A pair of pebbles is said to ``collide''
	if they fall on the same node or on two nodes that are neighbors. What is
	the expected number of pairs of pebbles that collide? About how large must
	m be before you would expect at least 1 pair of pebbles to collide?
\end{quote}

\section{Recurrence}

\begin{quote}
Consider the Recurrence $f(n) = 3f(n/2) + \sqrt(n)$
\end{quote}

\subsection{Use the Master method to solve this recurrence}



\subsection{Now use annihilators (and a transformation) to solve the recur- rence. Show
your work. (This is perhaps stating the obvious, but please note that your two
bounds should match)}




\section{Function}

\begin{quote}
\begin{lstlisting}
	int f (int n){
		if (n&=& 0) return 2;
		else if (n&=& 1) return 5;
		else{
			int val = 2*f (n-1);
			val = val - f (n-2);
			return val;
		}
	}
\end{lstlisting}
\end{quote}

\subsection{Write a recurrence relation for the value returned by f. Solve the
recurrence exactly. (Don’t forget to check it)}



\subsection{(b) Write a recurrence relation for the running time of f. Get a tight
upperbound (i.e. big-O) on the solution to this recurrence.}



\section{Silly-Sort }
\begin{lstlisting}
	Silly-Sort(A,i,j):
		if A[i] > A[j]:
			then exchange A[i] and A[j]
		if i+1 >= j:
			then return
		k = floor(j-i+1)/3)
		Silly-Sort(A,i,j-k)
		Silly-Sort(A,i+k,j)
		Silly-Sort(A,i,j-k)
	
\end{lstlisting}



\subsection{ Argue (by induction) that if n is the
length of A, then Silly- Sort(A,1,n)
correctly sorts the input array
$A[1\dots n]$}


\subsection{ Give a recurrence relation for the worst-case run time of Silly-Sort and a
tight bound on the worst-case run time}

\subsection{ Compare this worst-case runtime with that of insertion sort,merge
sort, heapsort and quicksort.}





\section{Primes and Probability}

\begin{quote}
	\textbf{
In this problem, you will use the following facts. 1) any integer can be
uniquely factored into primes; 2) the number of primes less than any number m
is $\Theta(m/ log m)$ (this is the prime number theorem). \\
We will also make use of the following notation for integers x and y: \\
1) $x|y$ means that x ``divides'' y, which means that there is no remainder
when you divide y by x. \\ 
and 2) $ x = y(modp)$ means that x and y have the same remainder when
divided by p, or in other words, $p|x−y$ \\
Show that for any integer x, x factors into at most log x primes. Hint: 2 is
the smallest prime.}
\end{quote}


\subsection{ Let x be some positive integer and let p be a prime chosen uniformly at
random from all primes less than or equal to m. Use the prime number theorem to
show that the probability that $p|x$ is $O\left( log x)(log m)/m\right)$. }

\subsection{ Now let x and y both be positive integers less than n and let p be a prime
chosen uniformly at random from all primes less than or equal to m. Using the
previous result, show that the probability that $ x ≡ y (mod p)$  is $ O\left( log
n)(log m)/m)\right)$ . }

\subsection{ If $ m = log^2 n$  in the previous problem, then what is the probability that
$x ≡ y (mod p)$. Hint: If you’re on the right track, you should be able to show
that this probability is ``small'', i.e. it goes to 0 as n gets large. }

\subsection{ Finally, show how to apply this result to the following problem. Alice and
Bob both have databases x and y where x and y have value no more than n, for n
a very large number (think terabytes). They want to check to see if their
databases are consistent (i.e. they want to check if they are the same) but
Alice does not want to have to send her entire database to Bob. What is an
algorithm Alice and Bob can use to check consistency with reasonably good
probability by sending a lot fewer bits? How many bits does Alice need to send
to Bob as a function of n, and what is the probability of failure, where
failure means that this algorithm says the databases are the same but in fact
they are different? }


\section{Bad Santa }
A child is presented with n boxes, one after another. Upon receiving
a box, the child must decide whether or not to open it. If the child does not
open a box, he is never allowed to revisit it. Half the
boxes have presents in them, but the decision about which boxes have presents
is made by an omniscient and malicious Santa who wants the child to open as
many empty boxes as possible before finding a present. \\ 
Devise and analyze a randomized algorithm for the child which minimizes the
expected number of boxes that need to be opened before the child finds the
first present. Assume Santa knows your algorithm, but can not predict the
random choices made by your algorithm. \\ 
Hint: Birthday paradox. \\
Note: This problem has applications to wireless networks: basically boxes are
time-steps, Santa is a jamming adversary, and opening a box means spending
energy to listen in a time-step.



\end{document}

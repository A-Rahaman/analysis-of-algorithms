%%%%%latex preamble%%%%%
\documentclass[titlepage]{article}\usepackage[]{graphicx}\usepackage[]{color}
%% maxwidth is the original width if it is less than linewidth
%% otherwise use linewidth (to make sure the graphics do not exceed the margin)
\makeatletter
\def\maxwidth{ %
  \ifdim\Gin@nat@width>\linewidth
  \linewidth
  \else
  \Gin@nat@width
  \fi
}
\makeatother


\usepackage{listings}
\definecolor{mygreen}{rgb}{0,0.6,0}
\definecolor{mygray}{rgb}{0.5,0.5,0.5}
\definecolor{mymauve}{rgb}{0.58,0,0.82}
\lstset{ %
  backgroundcolor=\color{white},   % choose the background color; you must add \usepackage{color} or \usepackage{xcolor}
  basicstyle=\footnotesize,        % the size of the fonts that are used for the code
  breakatwhitespace=false,         % sets if automatic breaks should only happen at whitespace
  breaklines=true,                 % sets automatic line breaking
  captionpos=b,                    % sets the caption-position to bottom
  commentstyle=\color{mygreen},    % comment style
  deletekeywords={...},            % if you want to delete keywords from the given language
  escapeinside={\%*}{*)},          % if you want to add LaTeX within your code
  extendedchars=true,              % lets you use non-ASCII characters; for 8-bits encodings only, does not work with UTF-8
  frame=single,                    % adds a frame around the code
  keepspaces=true,                 % keeps spaces in text, useful for keeping indentation of code (possibly needs columns=flexible)
  keywordstyle=\color{blue},       % keyword style
  language=Python,                 % the language of the code
  morekeywords={*,...},            % if you want to add more keywords to the set
  numbers=left,                    % where to put the line-numbers; possible values are (none, left, right)
  numbersep=5pt,                   % how far the line-numbers are from the code
  numberstyle=\tiny\color{mygray}, % the style that is used for the line-numbers
  rulecolor=\color{black},         % if not set, the frame-color may be changed on line-breaks within not-black text (e.g. comments (green here))
  showspaces=false,                % show spaces everywhere adding particular underscores; it overrides 'showstringspaces'
  showstringspaces=false,          % underline spaces within strings only
  showtabs=false,                  % show tabs within strings adding particular underscores
  stepnumber=2,                    % the step between two line-numbers. If it's 1, each line will be numbered
  stringstyle=\color{mymauve},     % string literal style
  tabsize=2,                       % sets default tabsize to 2 spaces
  title=\lstname                   % show the filename of files included with \lstinputlisting; also try caption instead of title
}
\usepackage{alltt}
\usepackage[sc]{mathpazo}
\usepackage{amsmath, amsthm, amssymb}
\usepackage{graphicx}
\usepackage[T1]{fontenc}
\usepackage{geometry}
\geometry{verbose,tmargin=2.5cm,bmargin=2.5cm,lmargin=1.5cm,rmargin=1.5cm}
\setcounter{secnumdepth}{2}
\setcounter{tocdepth}{2}
\usepackage{url}
\usepackage[unicode=true,pdfusetitle,
  bookmarks=true,bookmarksnumbered=true,bookmarksopen=true,bookmarksopenlevel=2,
breaklinks=false,pdfborder={0 0 1},backref=false,colorlinks=false]
{hyperref}
\hypersetup{pdfstartview={XYZ null null 1}}
\usepackage{float}
\usepackage{bm}
\usepackage{tikz}
 %changes default sectioning commands -> 1,a, etc.
%\usepackage{breakurl}
\renewcommand{\thesubsection}{(\alph{subsection})}
\renewcommand{\thesubsubsection}{\roman{subsection}.}
\usepackage{lastpage}
\usepackage{fancyhdr}
\pagestyle{fancy}

%%% Header and Footer %%% 
\lhead{}
\chead{\leftmark}
\rhead{}
\lfoot{Aaron Gonzales; Algorithms}
\cfoot{Homework 4}
\rfoot{Page \thepage\ of \pageref{LastPage}}
\IfFileExists{upquote.sty}{\usepackage{upquote}}{}

\begin{document}

\title{Homework 4 \\ CS561 Fall 2014}
\author{Aaron Gonzales}
\maketitle


\section{Activity Selection}
\begin{quote}
  \textbf{ Consider the following alternative greedy algorithms for the
	activity selection problem disdussed in class. For each algorithms, either
  prove or disprove that it constructs an optimal schedule. }
\end{quote}
\begin{enumerate}
  \item Choose an activity with shortest duration, discard all
	conflicting activities and recurse
\end{enumerate}
\subsubsection{answer:}

\begin{enumerate}
  \item Choose an activity that starts first, discard all conflicting
	activities and recurse
\end{enumerate}

\subsubsection{answer:}
\begin{enumerate}
  \item Choose an activity that ends latest, discard all conflicting
	activities and recurse
\end{enumerate}

\subsubsection{answer:}
\begin{enumerate}
  \item Choose an activity that ends latest, discard all
	conflicting activities and recurse
\end{enumerate}
\subsubsection{Answer:}

%%%%%%%%%%%%%%%%%%%%%%%%%%%%%%%%
\section{Weighted Activity Selection}
\begin{quote}
  \textbf{ Now consider a weighted version of the activity selection problem. Imagine
	that each activity, $a_i$ has a \textit{weight}, $w(a_i)$ (weights are
	totally unrelated to activity duration). Your goal is now to choose a set of
	non coinciting activites that give you the largest possible sum of weights,
  given an array of start times, end times, and values as input.}
\end{quote}

\subsection{Prove that the greedy algorithm described in class - Choose the
  activity that ends first and recurse - does not always return an optimal
schedule for this problem}
\subsubsection{Answer: }


\subsection{Describe an algorithm to compute the optimal schedule in $O(n^2)$
  time. Hint: 1) Sort the activities by finish times. 2) Let $m(j)$ be the
  maximum weight achievable from activites $a_1, a_2, \dots a_j$. 3) Come up
  with a recursive formulation for $m(j)$ and use dynamic programming. Hint 2:
  In the recursive formulation for $m(j)$ and use dynamic programming. Hint 2:
  In the recursion in step 3, it'll help if you precompute for each job $j$ the
  value $x_j$ which is the largest index $i < j$ such taht job $i$ is
  compatible with job $j$. Then when computing $m(j)$ consider that the optimal
schedule could either include job $j$ or not include job $j$. }

\subsubsection{Answer: }


\section{chessboards}

\begin{quote}
  \textbf{Consider the following problem: \\
	INPUT: Positive integers $r_1, \dots , r_n \text{and} c_1, \dots , c_n$. \\
	OUTPUT: An $n$ by $n$ matrix $A$ with 0/1 entries such that for all $i$ the
	sum of the ith row in A is $r_i$ and the sum of the ith column in A is
	$c_i$ if such a matrix exists. \\
	Think about the problem this way. You want to put pawns on an n by n
	chessboard so that the ith row has $r_i$ pawns and the ith column has $c_i$
	pawns. Consider the following greedy algorithm that constructs A row by
	row. Assume that the first $i - 1$ rows have been constructed. Let $a_j$ be
	the number of 1s in the jth column of in the first $i-1$ rows. Now the
	$r_i$ columns with maximum $c_j - a_j$ are assigned 1s in row i and thre
	rest of the columns are assigned 0s. That is, the columns that still need
	the most 1s are given 1s. Formally prove that this algorithm is correct
  using an exchange argument.}
\end{quote}


\subsubsection{Answer: }

\section{Hash Tables}
\begin{quote}
  \textbf{Suppose we can insert or delete an element into a hash table in
	$O(1)$ time. In order to ensure that our has h table is always big enough,
	without wasting a lot of memory, we will use the following global
  rebuilding rules:}
\end{quote}

\begin{itemize}
  \item \textbf{After an insertion, if the table is more than 3/4 full, we
	  allocate a new table twice as big as our current table, insert
	everything into the new table, and then free the old table.}
  \item \textbf{ after a deletion, if the table is less than 1/4 full, we
	  allocate a new table half as big as our current table, insert
	everythign into the new table, and then free the old table.}
\end{itemize}
\begin{quote}
  \textbf{Show that for any sequence of insertions and deletions, the
	amortized time per operation is still $O(1)$. Hint: Do not use
  potential functions.}
\end{quote}

\subsubsection{Answer: }



\section{some random data structure}
\begin{quote}
  \textbf{Suppose we are maintaining a data structure under a series of
  operations. Let $f(n)$ denote the actual running time of the nth operation.
For each of the following functions $f$, determine the resulting amortized cost
of a single operation.}
\end{quote}

\begin{itemize}
  \item \textbf{ $f(n) = n$ if n is a power of 2, and $f(n) = 1$ otherwise}
  \item \textbf{ $f(n) = n^2$ if n is a power of 2, and $f(n) = 1$ otherwise}
\end{itemize}

\subsubsection{Answer: }


\section{Extendable arrays}
\begin{quote}
  \textbf{An extendable array is a data structure that stores a sequence of
  items and supports the following operations.}
\end{quote}

\begin{itemize}
  \item \textbf{AddToFront(x) adds x to the beginning of the sequence.}
  \item \textbf{AddToEnd(x) adds x to the end of the sequence.}
  \item \textbf{LookUp(k) returns the kth item in the sequence or NULL if the
	current length of the sequence is less than k. }
\end{itemize}
\begin{quote}
  \textbf{Describe a simple data structre that implements an extendable array.
	Your AddToFront and AddToBack algorithms should take $O(1)$ amortized time
	and your LOOKUP algorithm should take $O(1)$ worst-case time. The data
	structure should use $O(n)$ space, where n is the current length of the
  sequence.}
\end{quote}

\subsubsection{Answer: }

\section{optimizing a data structure}
\begin{quote}
  \textbf{Describe and analyze a data structure to support the following
	operations on an array $A[1\dots n]$ as quickly as possible. Initially, $A[i]
  = 0 \forall i$.}
\end{quote}

\begin{itemize}
  \item \textbf{SetToOne(i) Given an index i such that $A[i] = 0$, set $A[i]$ to 1.}
  \item \textbf{GetValue(i) Given an index i, return $A[i]$. }
  \item \textbf{GetClosestRightZero(i) Given an index $i$, return the smallest
	index $j \geq i$ such that $A[j] = 0$ or report that no such index exists.}
\end{itemize}
\begin{quote}
  \textbf{The first two operations should run in worst-case constant time, and
  the amortized cost of the third operation should be as small as possible.}
\end{quote}

\subsubsection{Answer: }



\end{document}

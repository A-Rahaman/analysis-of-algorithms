\documentclass[11pt]{article}
%%
%% Jared's preamble.
%%

%
% Need these packages... 
%
\usepackage{latexsym}
\usepackage{amsmath}
\usepackage{amstext}
\usepackage{amsfonts}
\usepackage{amsopn}
\usepackage{float}

%
% Theorem like environments.
%
 \newtheorem{theorem}{Theorem}
 \newtheorem{lemma}[theorem]{Lemma}
 \newtheorem{corollary}[theorem]{Corollary}
 \newtheorem{proposition}[theorem]{Proposition}
 \newtheorem{fact}{Fact}
   \newtheorem{definition}{Definition}
   \newtheorem{observation}{Observation}


%algorithms and procedures
\floatstyle{ruled}
\newfloat{algorithm}{thp}{loa}
\floatname{algorithm}{Algorithm}

\floatstyle{ruled}
\newfloat{procedure}{thp}{lop}
\floatname{procedure}{Procedure}

\floatstyle{ruled}
\newfloat{protocol}{thp}{lop}
\floatname{protocol}{Protocol}



%
% Don't use this, `proof' and `proofby' redefine this env
% to allow you to specify a proof with a different type of 
% enclosing environment for the proof body.
%

% \newenvironment{proof}[1][Proof:]{\begin{trivlist}
% \item[\hskip \labelsep {\bfseries #1}]}{\end{trivlist} \qed}

%
% This is an environment that does nothing.  You could use this
% for a proof if you did not want an enclosing environment.
%
\newenvironment{none}{}{}



%
% often used in proofs.
%
\newcommand{\ih}{inductive hypothesis}


%
% BlackBoardBold symbols.
%
\newcommand{\integers}{\mathbb Z}
\newcommand{\reals}{\mathbb R}
\newcommand{\complex}{\mathbb C}
\newcommand{\rationals}{\mathbb Q}
\newcommand{\naturals}{\mathbb N}
\newcommand{\F}{\mathbb F}



%
% This adds the cute little box at the end of a proof.
% the proof and proofby environment automatically do this.
%
%\newcommand{\qed}{\nopagebreak\hfill $\Box$}


%
% this doesn't really work.
%
\newcommand*{\ubar}[1]{\ensuremath{\underline{#1}}}



%
% For referencing equations, lemmas, theorems etc.
%
\newcommand{\lref}[1]{Lemma~\ref{#1}}
\newcommand{\eqnref}[1]{Equation~\ref{#1}}
\newcommand{\thmref}[1]{Theorem~\ref{#1}}


%
% these are for text.  some people like to write iff or WLOG, so
% we will provide functions.  Remempler to use them like this:
% ``yadda yadda X \IFF\ Y yadda''
% or
% ``yadda \WLOG, yadda''
%
\newcommand{\IFF}{if and only if}
\newcommand{\WLOG}{without loss of generality}

% isomorphic to: this is =~
\newcommand{\iso}{\cong}

\newcommand{\defeq}{\stackrel{\mathrm{\small def}}{=}}

% derivitives.
\newcommand{\deriv}[2]{\tfrac{d #1 }{d #2}}
\newcommand{\partialderiv}[2]{\tfrac{\partial #1 }{\partial #2}}

%
% probability stuff.
%
\newcommand{\prob}[1]{\text{\bf Pr}\!\left[#1\right]}
\newcommand{\expect}[1]{\text{\bf E}\!\left[#1\right]}

%
% random functions of one argument.
%
\newcommand{\ceil}[1]{\left\lceil#1\right\rceil}
\newcommand{\floor}[1]{\left\lfloor#1\right\rfloor}
\newcommand{\abs}[1]{\left| #1 \right|}
\newcommand{\setsize}[1]{\left| #1 \right|}
\newcommand{\norm}[1]{\left\| #1 \right\|}
\newcommand{\pair}[1]{\left\langle #1 \right\rangle}
%
% random functions with no arguments.
%
\newcommand{\divides}{\mid}

%
% Such that is for equations not for text.
%
\newcommand{\suchthat}{\ :\ }


%
% quickly define math abreviations for repeated forms. e.g.
% \mathdef{\PIi}{\pi_i}
%
% which allows you to use $x_\PIi$ instead of $x_{\pi_i}$
%
\newcommand{\mathdef}[2]{\newcommand{#1}{\ensuremath{#2}}}

%
% New math operators.  These work like `gcd' or `log'.
%
\DeclareMathOperator{\lcm}{lcm}
\DeclareMathOperator{\img}{img}
\DeclareMathOperator{\GL}{GL}
\DeclareMathOperator{\argmax}{argmax}
\DeclareMathOperator{\avg}{avg}
%\DeclareMathOperator{\det}{det}

%
% the end
%

  
%\newcommand{\ans}[1]{\emph{Solution: #1}}

\newcommand{\ans}[1]{}

\begin{document}

\title{CS 361, HW4}

\author {Prof. Jared Saia, University of New Mexico}

\date{\emph{Due: February 17th, 2004}}
\maketitle

\begin{enumerate}

\item 
Consider the recurrence $T (n) = 2T (n/4) + n^{2}$
\begin{enumerate}
\item Use the recursion tree method to get a tight upper bound
(i.e. big-O) on the solution to this recurrence
\item Now use annihilators (and a transformation) to get a tight upper
bound on the solution to this recurrence.  Show your work.  (Note that
your two bounds should match)
\end{enumerate}

\item 
Consider the recurrence $T (n) = 2T (n/2) + \log^{2} n$
\begin{enumerate}
\item Use the Master method to get a general solution to this
recurrence.
\item Now use annihilators (and a transformation) to get a tight upper
bound on the solution to this recurrence.  Show your work.  (Note that
your two bounds should match)
\end{enumerate}

\item Consider the following function:
\begin{verbatim}
int f (int n){
  if (n==0) return 0;
  else if (n==1) return 1;
  else{
    int val = 6*f (n-1);
    val = val - 9*f (n-2);
    return val;
  }
}
\end{verbatim}

\begin{enumerate}
\item Write a recurrence relation for the \emph{value} returned by
$f$.  Solve the recurrence exactly.  (Don't forget to check it)
\item Write a recurrence relation for the \emph{running time} of $f$.
Get a tight upperbound (i.e. big-O) on the solution to this
recurrence. 
\end{enumerate}


\item Consider the following function:
\begin{verbatim}
int f (int n){
  if (n==0) return 0;
  else if (n==1) return 1;
  else{
    int val = 4*f (n-1);
    val = val - 4*f (n-2);
    return val;
  }
}
\end{verbatim}

\begin{enumerate}
\item Write a recurrence relation for the \emph{value} returned by
$f$.  Solve the recurrence exactly.  (Don't forget to check it)
\item Write a recurrence relation for the \emph{running time} of $f$.
Get a tight upperbound (i.e. big-O) on the solution to this
recurrence. 
\end{enumerate}

\end{enumerate}

\end{document}

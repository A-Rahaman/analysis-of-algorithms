%%%%%latex preamble%%%%%
\documentclass[titlepage]{article}\usepackage[]{graphicx}\usepackage[]{color}
%% maxwidth is the original width if it is less than linewidth
%% otherwise use linewidth (to make sure the graphics do not exceed the margin)
\makeatletter
\def\maxwidth{ %
  \ifdim\Gin@nat@width>\linewidth
  \linewidth
  \else
  \Gin@nat@width
  \fi
}
\makeatother


\usepackage{listings}
\definecolor{mygreen}{rgb}{0,0.6,0}
\definecolor{mygray}{rgb}{0.5,0.5,0.5}
\definecolor{mymauve}{rgb}{0.58,0,0.82}
\lstset{ %
  backgroundcolor=\color{white},   % choose the background color; you must add \usepackage{color} or \usepackage{xcolor}
  basicstyle=\footnotesize,        % the size of the fonts that are used for the code
  breakatwhitespace=false,         % sets if automatic breaks should only happen at whitespace
  breaklines=true,                 % sets automatic line breaking
  captionpos=b,                    % sets the caption-position to bottom
  commentstyle=\color{mygreen},    % comment style
  deletekeywords={...},            % if you want to delete keywords from the given language
  escapeinside={\%*}{*)},          % if you want to add LaTeX within your code
  extendedchars=true,              % lets you use non-ASCII characters; for 8-bits encodings only, does not work with UTF-8
  frame=single,                    % adds a frame around the code
  keepspaces=true,                 % keeps spaces in text, useful for keeping indentation of code (possibly needs columns=flexible)
  keywordstyle=\color{blue},       % keyword style
  language=Python,                 % the language of the code
  morekeywords={*,...},            % if you want to add more keywords to the set
  numbers=left,                    % where to put the line-numbers; possible values are (none, left, right)
  numbersep=5pt,                   % how far the line-numbers are from the code
  numberstyle=\tiny\color{mygray}, % the style that is used for the line-numbers
  rulecolor=\color{black},         % if not set, the frame-color may be changed on line-breaks within not-black text (e.g. comments (green here))
  showspaces=false,                % show spaces everywhere adding particular underscores; it overrides 'showstringspaces'
  showstringspaces=false,          % underline spaces within strings only
  showtabs=false,                  % show tabs within strings adding particular underscores
  stepnumber=2,                    % the step between two line-numbers. If it's 1, each line will be numbered
  stringstyle=\color{mymauve},     % string literal style
  tabsize=2,                       % sets default tabsize to 2 spaces
  title=\lstname                   % show the filename of files included with \lstinputlisting; also try caption instead of title
}
\usepackage{alltt}
\usepackage[sc]{mathpazo}
\usepackage{amsmath, amsthm, amssymb}
\usepackage{graphicx}
\usepackage[T1]{fontenc}
\usepackage{geometry}
\geometry{verbose,tmargin=2.5cm,bmargin=2.5cm,lmargin=1.5cm,rmargin=1.5cm}
\setcounter{secnumdepth}{2}
\setcounter{tocdepth}{2}
\usepackage{url}
\usepackage[unicode=true,pdfusetitle,
  bookmarks=true,bookmarksnumbered=true,bookmarksopen=true,bookmarksopenlevel=2,
breaklinks=false,pdfborder={0 0 1},backref=false,colorlinks=false]
{hyperref}
\hypersetup{pdfstartview={XYZ null null 1}}
\usepackage{float}
\usepackage{bm}
\usepackage{tikz}
 %changes default sectioning commands -> 1,a, etc.
%\usepackage{breakurl}
\renewcommand{\thesubsection}{(\alph{subsection})}
\renewcommand{\thesubsubsection}{\roman{subsection}.}
\usepackage{lastpage}
\usepackage{fancyhdr}
\pagestyle{fancy}

%%% Header and Footer %%% 
\lhead{}
\chead{\leftmark}
\rhead{}
\lfoot{Aaron Gonzales; Algorithms}
\cfoot{Homework 2}
\rfoot{Page \thepage\ of \pageref{LastPage}}
\IfFileExists{upquote.sty}{\usepackage{upquote}}{}

\begin{document}
\title{Homework 2, CS561, Fall 2014}
\author{Aaron Gonzales}
\maketitle


\section{Chip Testing }

\begin{quote}
\textbf{Professor Diogenes has $n$ supposedly identical integrated-circuit chips that
in principle are capable of testing each other. The professor's test jib
accommodates two chips at a time. When the jig is loaded, each chip tests the
other and reports whether it is good or bad. A good chip always reports
accurately whether the other chip is good or bad, but the professor cannot
trust the answer of a bad chip. Thus the following possible outcomes of a test
are as follows:}

\begin{table}[h]
  \begin{tabular}{lll}
	\hline
	\multicolumn{1}{|l|}{Chip A Says} & \multicolumn{1}{l|}{Chip B Says} & \multicolumn{1}{l|}{Conclusion} \\ \hline
	B is good                         & A is good                        & both are good, or both are bad  \\ \hline
	B is good                         & A is bad                         & at least one is bad             \\ \hline
	B is bad                          & A is good                        & at least one is bad             \\ \hline
	B is bad                          & A is bad                         & at least one is bad             \\ \hline
	\end{tabular}
  \end{table}

\end{quote}


\subsection{}
  \begin{quote}
\textbf{show that if more than $n/2$ chips are bad, the professor cannot
	necessarily determine which chips are good using any strategy based on this
	kind of pairwise test. Assume that the bad chips can conspire to fool the
	professor. }
  \end{quote}


  \subsection{ }
  \begin{quote}
  \textbf{consider the problem of finding a signle good chip from among $n$ chips,
	assuming that more than $n/2$ of the chips are good. Show that $\lfloor [n/2] \rfloor$ 
	pairwise tests are sufficient to reduce the problem to one of nearly
	half the size.}
  \end{quote}

  \subsection{}
  \begin{quote}
  \textbf{show that the good chips can be identified with $\Theta(n)$ pairwise tests,
	assuming that more than $n/2$ chips are good. Give and solve the recurrence
	that describes the number of tests. }
  \end{quote}



\section{Parenthesization }
\textbf{
  \begin{quote}
	Show via induction that a full Parenthesization of an $n$ element expression
	has exactly $n-1$ pairs of parenthesis. 
  \end{quote}
}

\section{h trees }
\begin{quote}
\textbf{ An h-tree is a rooted binary tree that is useful for designing self-healing
  networks (since they can be merged quickly). let $\ell$ be a postive integer.
  For $\ell$ a power of 2, the complete tree with $\ell$ leaf nodes is the
  unique h-tree with $\ell$ leaf nodes. For $\ell$ not a power of 2, a tree
  with $\ell$ leaf nodes is an h-tree if and only if (1) the root node, $r$,
  has two children; (2) the left subtree of $r$ is the root of a complete
  binary containing $2^{\lfloor log \ell \rfloor}$ leaf nodes; and (3) the
  right subtree of $r$ is an h-tree. Recall that a complete binary tree is one
  where every internal node has two children and every leaf node has the same
  depth. \\
Show the following by indcution:}

  \begin{enumerate}
	\item\textbf{ For all postive $\ell$ there is an unique h-tree with $\ell $ leaf
	  nodes.}
	\item \textbf{Call the h-tree with $\ell$ leaf nodes h-tree$(\ell)$. Then, the
	  height of h-tree$(\ell)$ is $\lceil log \ell \rceil$}.
  \end{enumerate}
\end{quote}

\section{Parenthesization - again }
\begin{quote}
 \textbf{ Find the optimal Parenthesization for a matrix-chain product whose sequence
  of dimensions is: $(3,2,4,1,2)$. (Don't forget to include the table used to
computer your result.)}
\end{quote}



\section{A bakery }
\begin{quote}
  \textbf{A bakery sells donuts in boxes of three different quantities, $x_1, x_2,
  x_3$. In the Donut Buying problem, you are given the numbers $x_1, x_2,
x_3$, and an integer $n$ and you should return either }
  \begin{enumerate}
	\item \textbf{the minimum number of boxes needed to obtain exactly $n$ donuts if
	this is possible, along with a set of boxes that obtains this minimum}
	  \item ``DOH'' if it is not possible to obtain exactly $n$ donuts. 
  \end{enumerate}
  \textbf{For example, if $x+1 = 4, x_2 =6, x_3 = 9, \text{and } n=17$, then you should
  return that 3 boxes suffices, with 2 boxes of size 4, and 1 box of size 9.
  However, if n=11, you should return DOH! since it is not possible to buy
exactly 11 donuts with these box sizes. }
\end{quote}

\subsection{For any positive $x$, let $m(x)$ be the minimum number of boxes
  needed to buy $x$ donuts if this is possible, or INFINITY $\infty$
  otherwise. Write a recurrence relation for the value of $m(x)$. Don't forget
the base case(s)!}



\subsection{ Give an efficient algorithm for solving Donut Buying. How does its
  running time dpeend on $x_1, x_2, x_3,\text{ and } n$? Is it an algorithm
that runs in polynomial time in the input sizes?} 




\section{Problem 15-5 (2nd) / 15-7 (3rd) Viterbi Algorthm. }
\begin{quote}
  \textbf{Not in this problem, a label can appear on more than one edge in the graph
  and can even appear on more than one edge leaving a given node in the graph.
  \\
  We can use dynamic programming on a directed graph $G = (V,E)$ for speech
  recognition. Each edge $(u,v) \subset E$ is labeled with a sound
  $\sigma(u,v)$ from a finite set $\Sigma$ of sounds. The labeldd graph is a
  form odel of a person speaking a restricted language. Each path in the graph
  starting from a distinguised vertex $v_0 \subset V$ corresponds to a possible
  sequence of sounds produced by the model. We define the label of a directed
path to be the concatenation of the labels of the edges on that path. }
\end{quote}
\subsection{ Describe an efficient algorithm that, given an edge-labeld graph
  $G$ with distinguised vertex $v_0$ and a sequence $s = \langle \sigma_1,
  \sigma_2, \dots, \sigma_k \rangle$ of sounds from $\Sigma$, returns a path in
  $G$ that begins at $v_0$ and has $s$ as its label, if any such path exists.
  Otherwise, the algorithm should return \textsc{no-such-path}. Analyze the
  running time of your algorithm. (Hint: you may find conceps from Chapter 22
useful.)}

  \begin{quote}
	\textbf{Now, suppose that every edge $(u,v) \subset E$ has an associated
	nonnegative probabilty $(p(u,v)$ of traversing the edge $(u,v)$ from
	vertex $u$ and thus producing the corresponding sound. The sum of the
	probabilities of the edges leaving any vertex equas 1. The probabilty of
	a path is defined to be the product of the probabilities of its edges. We
	can view the problem of a path beginning at $v_0$ as the probabilty that a
	``random walk'' beginning at $v_0$  will follow the specified path, where
	we randomly choose which edge to take leaving a vertex  $u$ according to
  the probabilities of the available edges leaving $u$. }
	\subsection{Extend your answer to part (a) so that if a path is returned,
	  it is a most probable path starting at $v_0$ and having label s. Analyze
	the running time of your algorithm. }
  \end{quote}





\section{The Chicken Brothers} 
\begin{quote}
Gus wants to open franchises of his restuarant, Los Pollos Hermanos,
along Central Avenue. There are $n$ possible locations for franchises where
location $i$ is at mile $i$ on Central. Each location $i>1$, is thus a distance
of 1 mile from the previous one. There are two rules.
\begin{itemize}
\item At each location there can be at most one restaurant, and the profit of a
  restaurant at location $i$ is $p_i$. 
  \item Any two restaurants must be at least 2 miles apart.
\end{itemize}
\end{quote}
\subsection{ Jesse proposed the following algorithm: Sort the locations by
  decreasing $p_i$ values, then greedily choose the next possible location,
  provided that it doesn't conflict with previously choses locations. Show that
Jesse's algorithm doesn't always give maximum profit.}



\subsection{Now consider a dynamic programming approach to this problem. For
  $i\geq 0 \text{let } m(i)$ be the maximum profit obtainable by using
  locations 1 throught $i$. Write a recurrence relation for $m(i)$. Don't
forget the base case(s).}


\subsection{Describe how you would create a dynamic program using the previous
recurrence. What is the run time of your algorithm?} 

\subsection{now Gus wants to solve a generalization of the problem. There are
  two changes. First, for $1 < i \leq n$, location $i$, is now distance $d_i$
  from location $i-1$. Second, any two restaurants must now be distance $k$
  appart for some parameter $k$. Write a new recurrence relation for this
problem. Don't forget the base case(s).}



\end{document}

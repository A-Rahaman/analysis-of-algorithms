%%%%%latex preamble%%%%%

\documentclass[titlepage]{article}\usepackage[]{graphicx}\usepackage[]{color}
%% maxwidth is the original width if it is less than linewidth
%% otherwise use linewidth (to make sure the graphics do not exceed the margin)
\makeatletter
\def\maxwidth{ %
  \ifdim\Gin@nat@width>\linewidth
  \linewidth
  \else
  \Gin@nat@width
  \fi
}
\makeatother


\usepackage{listings}
\lstset{language=Python} 
\definecolor{mygreen}{rgb}{0,0.6,0}
\definecolor{mygray}{rgb}{0.5,0.5,0.5}
\definecolor{mymauve}{rgb}{0.58,0,0.82}
\lstset{ %
  backgroundcolor=\color{white},   % choose the background color; you must add \usepackage{color} or \usepackage{xcolor}
  basicstyle=\footnotesize,        % the size of the fonts that are used for the code
  breakatwhitespace=false,         % sets if automatic breaks should only happen at whitespace
  breaklines=true,                 % sets automatic line breaking
  captionpos=b,                    % sets the caption-position to bottom
  commentstyle=\color{mygreen},    % comment style
  deletekeywords={...},            % if you want to delete keywords from the given language
  escapeinside={\%*}{*)},          % if you want to add LaTeX within your code
  extendedchars=true,              % lets you use non-ASCII characters; for 8-bits encodings only, does not work with UTF-8
  frame=single,                    % adds a frame around the code
  keepspaces=true,                 % keeps spaces in text, useful for keeping indentation of code (possibly needs columns=flexible)
  keywordstyle=\color{blue},       % keyword style
  language=Octave,                 % the language of the code
  morekeywords={*,...},            % if you want to add more keywords to the set
  numbers=left,                    % where to put the line-numbers; possible values are (none, left, right)
  numbersep=5pt,                   % how far the line-numbers are from the code
  numberstyle=\tiny\color{mygray}, % the style that is used for the line-numbers
  rulecolor=\color{black},         % if not set, the frame-color may be changed on line-breaks within not-black text (e.g. comments (green here))
  showspaces=false,                % show spaces everywhere adding particular underscores; it overrides 'showstringspaces'
  showstringspaces=false,          % underline spaces within strings only
  showtabs=false,                  % show tabs within strings adding particular underscores
  stepnumber=2,                    % the step between two line-numbers. If it's 1, each line will be numbered
  stringstyle=\color{mymauve},     % string literal style
  tabsize=2,                       % sets default tabsize to 2 spaces
  title=\lstname                   % show the filename of files included with \lstinputlisting; also try caption instead of title
}
\usepackage{alltt}
\usepackage[sc]{mathpazo}
\usepackage{amsmath, amsthm, amssymb}
\usepackage{graphicx}
\usepackage[T1]{fontenc}
\usepackage{geometry}
\geometry{verbose,tmargin=2.5cm,bmargin=2.5cm,lmargin=1.5cm,rmargin=1.5cm}
\setcounter{secnumdepth}{2}
\setcounter{tocdepth}{2}
\usepackage{url}
\usepackage[unicode=true,pdfusetitle,
  bookmarks=true,bookmarksnumbered=true,bookmarksopen=true,bookmarksopenlevel=2,
breaklinks=false,pdfborder={0 0 1},backref=false,colorlinks=false]
{hyperref}
\hypersetup{pdfstartview={XYZ null null 1}}
\usepackage{float}
\usepackage{bm}
\usepackage{tikz}
 %changes default sectioning commands 
%\usepackage{breakurl}
\renewcommand{\thesubsection}{(\alph{subsection})}
\renewcommand{\thesubsubsection}{\roman{subsection}.}
\usepackage{lastpage}
\usepackage{fancyhdr}
\pagestyle{fancy}
\lhead{}
\chead{\leftmark}
\rhead{}
\lfoot{Aaron Gonzales; Algorithms}
\cfoot{Homework 1}
\rfoot{Page \thepage\ of \pageref{LastPage}}
\IfFileExists{upquote.sty}{\usepackage{upquote}}{}

\begin{document}

\title{Homework 1, CS561, Fall 2014}
\author{Aaron Gonzales}
\maketitle

%%%%%%%%knitr option%%%%%%%



\section{ exercise 3.1-5 CLRS - Prove theorem 3.1}


\section{ Let $f(n) and g(n)$ be two functions that take on nonnegative values and
assume $f(n) = O(g(n))$. Prove that $g(n) =\Omega(f(n))$.}

By definition, $\Omega(g(n)) = \{ f(n) : $ there exist positive constants $c, n_0$
such that $0 \leq cg(n) \leq f(n) \forall  n \geq n_0\}$.

By definition, $O(g(n)) = \{ f(n) : $ there exist positive constants $c, n_0$
such that $0 \leq f(n) \leq cg(n) \forall  n \geq n_0\}$.

if $f(n)$ is bounded above by $g(n)$, $g(n)$ must be at least equal to or
lower than $f(n)$ for all $n \geq n_0$.





\section{ Problem 3-2 on Relative asymptotic growths}
\begin{table}[h]
  \begin{tabular}{ll|l|l|l|l|l}
	A & B  & $O$  & $o$  & $\Omega$  & $\omega$  & $\Theta$  \\ \hline	
	$lg^k $&  $n^\epsilon$   &  &  &  &  &  \\ \hline	
	$n^k$     & $c^n$  &  &  &  &  &  \\ \hline	
	$\sqrt{n}$ & $n^{sin n}$     &  &  &  &  &  \\ \hline	
	$2^n$      & $2^{n/2}$  &  &  &  &  &  \\ \hline	
	$n^{lg c}$     & $c^{lg n}$  &  &  &  &  &  \\ \hline	
	$lg(n!)$     &  $lg(n^n)$ &  &  &  &  &  
  \end{tabular}
\end{table}


\section{ Assume you have functions $f$ and $g$, such that $f(n)$ is $O(g(n))$.
  for each of the following statements, decide wether you think it is true or false and give
either a proof or a counterexample. }

\subsection{ $log_2f(n) = O(log_2(g(n))$}

counterexample: f(2) g(1).


\subsection{ $2^{ f(n) } = O(2^{ g(n) })$}
False.
if $f(n) = 2n$ and 

\subsection{$  f(n)^2 = O(g(n)^2)$}

True. 

$O$ is a loose upper bound. $0 \leq f(n) \leq g(n)$. 

\section{ Problem 7-3: Alternative Quicksort analysis}
\subsection{ 
Argue that, given an array of size $n$, the probability that any particular element
is chosen as the pivot is $1/n$. Use this to define indicator random variables
$X_i = I\{ i$ th smallest element is chosen as the pivot$\}$. What is
$E[X_i]$?}

$Pr(X=X_i) = 1/n$

Defining indicator variable $X_i \left[_{0: not pivot}^{1: if pivot}\right.]$

\[ \sum_{1}^{n} E(X_i) = X_i Pr(x=X_i) + X_2Pr(x = X_2) + \dots + X_n
Pr(x=X_n) \]



\subsection{}
Let $T(n)$ be a random variable denoting the running time of quicksort on an
array of size $n$. Argue that \\
\[ E[T(n)] = E \left[ \sum_{q=1}^{n} X_q (T(q-1)) + T(n-q) + \Theta(n))\right] \]

The basic quicksort recurrence is taken from equation 7.1:

\[ T(n) = T(Q) + (T(n-q-1) + \Theta(n) \]

if we define 
\[ Q' = q +1 \]
we can rewrite the recurrence as 

\[ T(n) = T(Q+1) + (T(n-Q') + \Theta(n) \]

and with our indicator variables...

\[ T(n) = \sum_{Q' = 1}^{n} \left( T(Q+1) + (T(n-Q') + \Theta(n) \right) \]

\[ E\left[T(n)\right] = E\left[ \sum_{Q' = 1}^{n} X_q \left( T(Q+1) + (T(n-Q') + \Theta(n) \right) \right] \]



\subsection{}
Show that we can rewrite equation 7.5 as:

\[ E[T(n) ] = \frac{2}{n} \sum_{q=2}^{n-1} E[T(q)] + \Theta(n) \].


\[ E[T(n)] = E \left[ \sum_{q=1}^{n} X_q (T(q-1)) + T(n-q) + \Theta(n))\right] \]

by LOE, we get

\[ E[T(n)] = E \left[ X_q \sum_{q=1}^{n} (T(q-1)) + T(n-q) + \Theta(n))\right] \]



\subsection{}
show that 
\[ \sum_{k=2}^{n-1}k\, lg\, k \leq \frac{1}{2} n^2\, lg\, n - \frac{1}{8} n^2 \].

\subsection{}

Using the bound from equation 7.7, show that the recurrance in equation 7.6 has
the solution $E[T(n)] = \Theta(n lg n)$. 




\section{ Ladders }
  Imagine you are doing a stress test on a particular model of smart phones. you
  have a ladder with $n$ rungs. You want ot determine the highest rung from which
  you cand rop a phone wihtout it breaking and you want to do it iwth the
smallest number of phone drops. 

\subsection{ Imagine that you have exactly 2 phones. Devise an algorithm that
  can determine
the highest safe rung using $o(n)$ drops. (little o). }

  Let it be stated that a ladder must have at least one one rung. If it has
  one rung, we only need one phone to test and see if that rung is safe (if it
  breaks, it's unsafe, if not, it's safe).

  If a ladder has $n > 1$ rungs, we start by dropping the phone from the middle
  ladder rung ($n/2$). if the phone breaks, we search the bottom half of the
  ladder iteratively (first rung, second rung, etc.). If it doesn't break, we
  iteratively search the upper portion of the ladder. 

  We only test at most $\frac{n}{2}$ rungs, which $\frac{n}{2} < n
  \forall n \geq 2$. 
  By definition, $o(g(n)) = \{ f(n) : $ for any positive constant $c>0$, there
  exisists a constant $n_0 > 0$ such that $0 \leq f(n) \leq cg(n) \, \forall \,
  n \, \geq n_0\}$.
  as such, the algorithm is $o(n)$. 

\subsection{Now suppose you have $k$ phones. Devise an algorithm that can
  determine the highest safe rung with the smallest number of drops. If
  $f_k(n)$
  is the number of drops that your algorithm needs, what is 
  $f_k(n)$ asymptotically? Hint: you shoudl ensure that 
  $f_{k+1}(n) = o(f_k(n))$ for any $k$.}

  This is an example of binary search. If we assume that the rungs in the
  ladder are in a sorted order (and how could they not be?) the algorithm could
  be stated as follows:

  As before, start with the middle rung of the ladder ($\frac{n}{2}$). If the
  phone dropped breaks, do the same procedure on the bottom half of the ladder.
  If it doesn't break, do the same procedure on the upper half of the ladder.
  each recursive step reduces the size of our search space by $\frac{1}{2}$. If
  for some reason we get to $k=1$, then iteratively search the current subarray
  and return the rung prior to the rung on which the phone broke.

  %Proof by induction that $f_k(n) = O(log n)$ follows:
  %Let $P(n)$ be the assertion that our binary search of the ladder's rungs
  %works correctly for a ladder of size $n$. 

  %%Our base case is a ladder of size 1: If the ladder has 1 rung, we only need
  %one drop to determine safety. 
  %If $P(n)$ works for all ladders of size 1, then it should work for a ladder of size $n+1$. 
%
  %we have two real cases here:
  %Case 1: the rung $n+1$ is not safe (phone breaks)
  	%
  %Case 2: the rung $n+1$ is safe (phone breaks)
%
  %Given that our binary search will always reduce the size of the search space
  %down to one rung, it will always tell us if 

  we can state this recurrence relation as:

  \[ T(n) = T(n/2) + C \]

  And using the master theorem, where 

$ Case 2 => T(n) = O(n^[logb a] * (log n)^(k+1)) $ 
%O(n^[logb a] * (log n)^(k+1))

%We already showed that n ^(logb a) in our case is n^0, which is 1, so we can make this:

%O(1 * (log n)^(k+1))

%and of course 1 is the multiplicative identity so:

%O((log n)^(k+1))

%and we're using 0 for our little k, so:

%O((log n)^1)

%and 1 is the power identity:

%O(log n)





  \section{ The game of Match. }



\subsection{ If we shuffle the deck and turn over three cards, what is the
probability that they form a match? Hint: given the first two
cards, what is the probability that the third forms a match?}

We know that given any two cards, only one more card can make a match. If we
draw two cards, only one more card in the deck can make a match for the prior
two cards, so we get out of our deck of $n=27$ cards, $\frac{1}{25}$.

\subsection{ If we shuffle the deck and turn over n cards where $n ≤ 27$, what
is the expected number of matches, where we count each match
separately even if they overlap? Note: The cards in a match do
not need to be adjacent! Is your expression correct for $n = 27$?}







  \section{ Drunken Debutants}



  \section{ Pairs on a circle}


  \end{document}

%%%%%latex preamble%%%%%

\documentclass[titlepage]{article}\usepackage[]{graphicx}\usepackage[]{color}
%% maxwidth is the original width if it is less than linewidth
%% otherwise use linewidth (to make sure the graphics do not exceed the margin)
\makeatletter
\def\maxwidth{ %
  \ifdim\Gin@nat@width>\linewidth
  \linewidth
  \else
  \Gin@nat@width
  \fi
}
\makeatother


\usepackage{listings}
\lstset{language=Python} 
\definecolor{mygreen}{rgb}{0,0.6,0}
\definecolor{mygray}{rgb}{0.5,0.5,0.5}
\definecolor{mymauve}{rgb}{0.58,0,0.82}
\lstset{ %
  backgroundcolor=\color{white},   % choose the background color; you must add \usepackage{color} or \usepackage{xcolor}
  basicstyle=\footnotesize,        % the size of the fonts that are used for the code
  breakatwhitespace=false,         % sets if automatic breaks should only happen at whitespace
  breaklines=true,                 % sets automatic line breaking
  captionpos=b,                    % sets the caption-position to bottom
  commentstyle=\color{mygreen},    % comment style
  deletekeywords={...},            % if you want to delete keywords from the given language
  escapeinside={\%*}{*)},          % if you want to add LaTeX within your code
  extendedchars=true,              % lets you use non-ASCII characters; for 8-bits encodings only, does not work with UTF-8
  frame=single,                    % adds a frame around the code
  keepspaces=true,                 % keeps spaces in text, useful for keeping indentation of code (possibly needs columns=flexible)
  keywordstyle=\color{blue},       % keyword style
  language=Octave,                 % the language of the code
  morekeywords={*,...},            % if you want to add more keywords to the set
  numbers=left,                    % where to put the line-numbers; possible values are (none, left, right)
  numbersep=5pt,                   % how far the line-numbers are from the code
  numberstyle=\tiny\color{mygray}, % the style that is used for the line-numbers
  rulecolor=\color{black},         % if not set, the frame-color may be changed on line-breaks within not-black text (e.g. comments (green here))
  showspaces=false,                % show spaces everywhere adding particular underscores; it overrides 'showstringspaces'
  showstringspaces=false,          % underline spaces within strings only
  showtabs=false,                  % show tabs within strings adding particular underscores
  stepnumber=2,                    % the step between two line-numbers. If it's 1, each line will be numbered
  stringstyle=\color{mymauve},     % string literal style
  tabsize=2,                       % sets default tabsize to 2 spaces
  title=\lstname                   % show the filename of files included with \lstinputlisting; also try caption instead of title
}
\usepackage{alltt}
\usepackage[sc]{mathpazo}
\usepackage{amsmath, amsthm, amssymb}
\usepackage{graphicx}
\usepackage[T1]{fontenc}
\usepackage{geometry}
\geometry{verbose,tmargin=2.5cm,bmargin=2.5cm,lmargin=1.5cm,rmargin=1.5cm}
\setcounter{secnumdepth}{2}
\setcounter{tocdepth}{2}
\usepackage{url}
\usepackage[unicode=true,pdfusetitle,
  bookmarks=true,bookmarksnumbered=true,bookmarksopen=true,bookmarksopenlevel=2,
breaklinks=false,pdfborder={0 0 1},backref=false,colorlinks=false]
{hyperref}
\hypersetup{pdfstartview={XYZ null null 1}}
\usepackage{float}
\usepackage{bm}
\usepackage{tikz}
 %changes default sectioning commands 
%\usepackage{breakurl}
\renewcommand{\thesubsection}{(\alph{subsection})}
\renewcommand{\thesubsubsection}{\roman{subsection}.}
\usepackage{lastpage}
\usepackage{fancyhdr}
\pagestyle{fancy}
\lhead{}
\chead{\leftmark}
\rhead{}
\lfoot{Aaron Gonzales; Algorithms}
\cfoot{Homework 1}
\rfoot{Page \thepage\ of \pageref{LastPage}}
\IfFileExists{upquote.sty}{\usepackage{upquote}}{}

\begin{document}

\title{Homework 1, CS561, Fall 2014}
\author{Aaron Gonzales}
\maketitle

%%%%%%%%knitr option%%%%%%%



\section{ exercise 3.1-5 CLRS - Prove theorem 3.1}
For any two functions $f(n)$ and $g(n)$, we have $f(n) = \Theta(g(n))$ if and
only if $f(n) = O(g(n))$ and $f(n) = \Omega(g(n))$.

By definition, $\Omega(g(n)) = \{ f(n) : $ there exist positive constants $c, n_0$
such that $0 \leq cg(n) \leq f(n) \forall  n \geq n_0\}$.

By definition, $O(g(n)) = \{ f(n) : $ there exist positive constants $c, n_0$
such that $0 \leq f(n) \leq cg(n) \forall  n \geq n_0\}$.

Let f and g be functions, such that $f(n)=\Theta(g(n))$.

By definition of $\Theta$, there exist positive constants $k1$ and $k2$ such
that, for sufficiently large $n$:
\[ k_1 * g(n)\leq f(n) \leq k_2* g(n) \]
Thus, for sufficiently large n:
\[ f(n) \leq k_2*g(n) \]
Therefore $f(n)=O(g(n))$ by definition.

And, for sufficiently large n:
\[ k_1*g(n)\leq f(n)\]
Therefore $f(n)=\Omega(g(n))$ by definition.


\section{ Let $f(n) and g(n)$ be two functions that take on nonnegative values and
assume $f(n) = O(g(n))$. Prove that $g(n) =\Omega(f(n))$.}

By definition, $\Omega(g(n)) = \{ f(n) : $ there exist positive constants $c, n_0$
such that $0 \leq cg(n) \leq f(n) \forall  n \geq n_0\}$.

By definition, $O(g(n)) = \{ f(n) : $ there exist positive constants $c, n_0$
such that $0 \leq f(n) \leq cg(n) \forall  n \geq n_0\}$.

if $f(n)$ is bounded above by $g(n)$, $g(n)$ must be at least equal to or
lower than $f(n)$ for all $n \geq n_0$.





\section{ Problem 3-2 on Relative asymptotic growths}
\begin{table}[h]
  \begin{tabular}{ll|l|l|l|l|l}
	A & B  & $O$  & $o$  & $\Omega$  & $\omega$  & $\Theta$  \\ \hline	
	$lg^k\,n$ &  $n^\epsilon$   & F  & F  & T   &T  &F  \\ \hline	
	$n^k$     & $c^n$  & T & T  &F  &F  &F  \\ \hline	
	$\sqrt{n}$ & $n^{sin n}$  & F  &  F &  F &  F &  F \\ \hline	
	$2^n$  & $2^{n/2}$   & T &F  & T  & F  & T  \\ \hline	
	$n^{lg c}$ & $c^{lg n}$ & T  & F & T  & F  & T  \\ \hline	
	$lg(n!)$ & $lg(n^n)$ &T   & F  &T  &F  & T  
  \end{tabular}
\end{table}


\section{ True or false }
\begin{quote}
	\bf{Assume you have functions $f$ and $g$, such that $f(n)$ is $O(g(n))$.
  for each of the following statements, decide wether you think it is true or false and give
either a proof or a counterexample. }
\end{quote}

\subsection{ $log_2f(n) = O(log_2(g(n))$}
False only in the following case: $f(2),\, g(1)$.
\[ f(2) = log_2(2) = 1 \]
\[ g(1) = log_2(1) = 0 \]
For large enough $n$, this is true, as this would become $O(log_2(n)) =
O(log_2(2))$. 

\subsection{ $2^{ f(n) } = O(2^{ g(n) })$}
False.
if $f(n) = 2n$ and 


\subsection{$  f(n)^2 = O(g(n)^2)$}

For all $n$ we can have $f(n)^2 = g(n)^2$.



True. 

$O$ is a loose upper bound. $0 \leq f(n) \leq g(n)$. 

\section{ Problem 7-3: Alternative Quicksort analysis}
\subsection{ 
Argue that, given an array of size $n$, the probability that any particular element
is chosen as the pivot is $1/n$. Use this to define indicator random variables
$X_i = I\{ i$ th smallest element is chosen as the pivot$\}$. What is
$E[X_i]$?}

$Pr(X=X_i) = 1/n$

Defining indicator variable $X_i \left[_{0: not pivot}^{1: if pivot}\right.]$

\[ \sum_{1}^{n} E(X_i) = X_i Pr(x=X_i) + X_2Pr(x = X_2) + \dots + X_n
Pr(x=X_n) \]



\subsection{}
Let $T(n)$ be a random variable denoting the running time of quicksort on an
array of size $n$. Argue that \\
\[ E[T(n)] = E \left[ \sum_{q=1}^{n} X_q (T(q-1)) + T(n-q) + \Theta(n))\right] \]

The basic quicksort recurrence is taken from equation 7.1:

\[ T(n) = T(Q) + (T(n-q-1) + \Theta(n) \]

% if we define 
% \[ Q' = Q + 1 \]
%we can rewrite the recurrence as 

%[ T(n) = T(Q+1) + (T(n-Q') + \Theta(n) \]

%nd with our indicator variables...

%[ T(n) = \sum_{Q' = 1}^{n} \left( T(Q+1) + (T(n-Q') + \Theta(n) \right) \]

%[ E\left[T(n)\right] = E\left[ \sum_{Q' = 1}^{n} X_q \left( T(Q+1) + (T(n-Q') + \Theta(n) \right) \right] \]



\subsection{}
Show that we can rewrite equation 7.5 as:

\[ E[T(n) ] = \frac{2}{n} \sum_{q=2}^{n-1} E[T(q)] + \Theta(n) \].


\[ E[T(n)] = E \left[ \sum_{q=1}^{n} X_q (T(q-1)) + T(n-q) + \Theta(n))\right] \]

by LOE, we get

\[ E[T(n)] = E \left[ X_q \sum_{q=1}^{n} (T(q-1)) + T(n-q) + \Theta(n))\right] \]



\subsection{}
show that 
\[ \sum_{k=2}^{n-1}k\, lg\, k \leq \frac{1}{2} n^2\, lg\, n - \frac{1}{8} n^2 \].

\subsection{}

Using the bound from equation 7.7, show that the recurrance in equation 7.6 has
the solution $E[T(n)] = \Theta(n lg n)$. 
Comparison-based sorting is $\Omega(n lg n)$, and for a function to be
$\Theta(n lg n)$ it must be both $\Omega(n lg n)$ and $\Omega(n lg n)$, which
this is. 




\section{ Ladders }
  Imagine you are doing a stress test on a particular model of smart phones. you
  have a ladder with $n$ rungs. You want ot determine the highest rung from which
  you cand rop a phone wihtout it breaking and you want to do it with the
smallest number of phone drops. 

\subsection{ Imagine that you have exactly 2 phones. Devise an algorithm that
  can determine
the highest safe rung using $o(n)$ drops. (little o). }

  Let it be stated that a ladder must have at least one one rung. If it has
  one rung, we only need one phone to test and see if that rung is safe (if it
  breaks, it's unsafe, if not, it's safe).

  If a ladder has $n > 1$ rungs, we start by dropping the phone from the middle
  ladder rung ($\frac{n}{2}$). if the phone breaks, we search the bottom half of the
  ladder iteratively (first rung, second rung, etc.). If it doesn't break, we
  iteratively search the upper portion of the ladder. 

  We only test at most $\frac{n}{2}$ rungs, which $\frac{n}{2} < n
  \forall n \geq 2$. 
  By definition, $o(g(n)) = \{ f(n) : $ for any positive constant $c>0$, there
  exisists a constant $n_0 > 0$ such that $0 \leq f(n) \leq cg(n) \, \forall \,
  n \, \geq n_0\}$.
  as such, the algorithm is $o(n)$. 

\subsection{Now suppose you have $k$ phones. Devise an algorithm that can
  determine the highest safe rung with the smallest number of drops. If
  $f_k(n)$
  is the number of drops that your algorithm needs, what is 
  $f_k(n)$ asymptotically? Hint: you should ensure that 
  $f_{k+1}(n) = o(f_k(n))$ for any $k$.}

  This is an example of binary search. If we assume that the rungs in the
  ladder are in a sorted order (and how could they not be? did my adversary
  misnumber my rungs?)  the algorithm could
  be stated as follows:

  As before, start with the middle rung of the ladder ($\frac{n}{2}$). (If the
  ladder has an even number of rungs, choose the $\lfloor
  \frac{n}{2} \rfloor$ rung). If the
  phone dropped breaks, do the same procedure on the bottom half of the ladder.
  If it doesn't break, do the same procedure on the upper half of the ladder.
  each recursive step reduces the size of our search space by $\frac{1}{2}$. If
  for some reason we get to $k=1$, then iteratively search the current subarray
  and return the rung prior to the rung on which the phone broke.

  we can state this recurrence relation as:

  \[ T(n) = T(n/2) + C \]
  which is $\Theta(log\,n)$. Since our binary search is $\Theta(log\,n)$, we
  get our work being done by the algorithm as $o(n)$. 
  

  \section{ The game of Match. }
  \begin{quote}
  The game of Match is played with a special deck of 27 cards. Each card has
  three attributes: color, shape and number. The possible color values are
  \{red, blue, green\}, the possible shape values are \{square, circle, heart\},
  and the possible number values are \{1, 2, 3\}. Each of the $3 ∗ 3 ∗ 3 = 27$
  possible combinations is represented by a card in the deck. A match is a set
  of 3 cards with the property that for every one of the three attributes,
  either all the cards have the same value for that attribute or they all have
  different values for that attribute. For example, the following three cards
  are a match: (3, red, square), (2, blue, square), (1, green, square).
  \end{quote}

\subsection{ If we shuffle the deck and turn over three cards, what is the
probability that they form a match? Hint: given the first two
cards, what is the probability that the third forms a match?}

We know that given any two cards, only one more card can make a match. If we
draw two cards, only one more card in the deck can make a match for the prior
two cards. After drawing two cards, we only have 25 cards left from which we
can draw, so we get $Pr(match) = \frac{1}{25}$. 

\subsection{ If we shuffle the deck and turn over $n$ cards where $n \leq 27$, what
is the expected number of matches, where we count each match
separately even if they overlap? Note: The cards in a match do
not need to be adjacent! Is your expression correct for $n = 27$?}

I believe we can have ${n \choose 3} * \frac{1}{25}$ possible matches, which
comes to 

\begin{align*}
	Total Matches &={n \choose 3} * \frac{1}{25} \\
              &= {27 \choose 3}   * \frac{1}{25} \\
              &=   \frac{27!}{3!(27-3)!} * \frac{1}{25} \\
				&= \frac{27 * 26 * 25}{6} * \frac{1}{25}  \\
				&= 117
\end{align*}



  \section{ Drunken Debutants}
Drunken Debutants: Imagine that there are n debutants, each with her own
porsche. After a late and wild party, each debutante stumbles into a porsche
selected independently and uniformly at random (thus, more than one debutant
may wind up in a porsche). Let X be a random variable giving the number of
debutants that wind up in their own porsche. Use linearity of expectation to
compute the expected value of X. Now use Markov’s inequality, to bound the
probability that X is larger than k for any positive k.

We have $n$ debutants and $n$ porsches. 
$\frac{1}{n}$ is our probability of a debutant getting into her own porche.
Let $X_i$ be an indicator variable saying 

\[
	X_i = I[X]
  \begin{cases}
   1 & \text{if debutante gets in her own car} \\
   0       & \text{if not } 
  \end{cases}
\]

The number of debutants who return to their own can be expressed as:
\[ X_i = \sum_{i=1}^n X_i \]

and the expected number of debutants can be expressed as:

\[ E [X_i] = E \left[ \sum_{i=1}^n X_i \right] \]

By linearity of expectation we can state it as such:

\begin{align*}
	E [X_i] &= E \left[ \sum_{i=1}^n X_i \right] \\
	E [X_i] &=  \sum_{i=1}^n E[X_i] \\ 
	&=  \sum_{i=1}^n \frac{1}{n} \\ 
	&=  ln\, n \\ 
\end{align*}


Markov's theorem states:
\[ \mathbf{Pr} \left[ X \geq t \right] \leq \frac{\mathbf[R]}{t} \]



  \section{ Pairs on a circle}
Imagine n points are distributed uniformly at random on the perimeter of a
circle that has circumference 1. Show that the expected number of pairs of
points that are within distance $\Theta(1/n^2)$ of each other is greater than 1. FYI:
this problem has applications in efficient routing in peer-to-peer networks.
Hint: Partition the circle into $n^2/k$ regions of size $k/n^2$ for some constant
k; then use the Birthday paradox to solve for the necessary k.

  \end{document}

%%%%%latex preamble%%%%%

\documentclass[titlepage]{article}\usepackage[]{graphicx}\usepackage[]{color}
%% maxwidth is the original width if it is less than linewidth
%% otherwise use linewidth (to make sure the graphics do not exceed the margin)
\makeatletter
\def\maxwidth{ %
  \ifdim\Gin@nat@width>\linewidth
  \linewidth
  \else
  \Gin@nat@width
  \fi
}
\makeatother


\usepackage{listings}
\lstset{language=Python} 
\definecolor{mygreen}{rgb}{0,0.6,0}
\definecolor{mygray}{rgb}{0.5,0.5,0.5}
\definecolor{mymauve}{rgb}{0.58,0,0.82}
\lstset{ %
  backgroundcolor=\color{white},   % choose the background color; you must add \usepackage{color} or \usepackage{xcolor}
  basicstyle=\footnotesize,        % the size of the fonts that are used for the code
  breakatwhitespace=false,         % sets if automatic breaks should only happen at whitespace
  breaklines=true,                 % sets automatic line breaking
  captionpos=b,                    % sets the caption-position to bottom
  commentstyle=\color{mygreen},    % comment style
  deletekeywords={...},            % if you want to delete keywords from the given language
  escapeinside={\%*}{*)},          % if you want to add LaTeX within your code
  extendedchars=true,              % lets you use non-ASCII characters; for 8-bits encodings only, does not work with UTF-8
  frame=single,                    % adds a frame around the code
  keepspaces=true,                 % keeps spaces in text, useful for keeping indentation of code (possibly needs columns=flexible)
  keywordstyle=\color{blue},       % keyword style
  language=Octave,                 % the language of the code
  morekeywords={*,...},            % if you want to add more keywords to the set
  numbers=left,                    % where to put the line-numbers; possible values are (none, left, right)
  numbersep=5pt,                   % how far the line-numbers are from the code
  numberstyle=\tiny\color{mygray}, % the style that is used for the line-numbers
  rulecolor=\color{black},         % if not set, the frame-color may be changed on line-breaks within not-black text (e.g. comments (green here))
  showspaces=false,                % show spaces everywhere adding particular underscores; it overrides 'showstringspaces'
  showstringspaces=false,          % underline spaces within strings only
  showtabs=false,                  % show tabs within strings adding particular underscores
  stepnumber=2,                    % the step between two line-numbers. If it's 1, each line will be numbered
  stringstyle=\color{mymauve},     % string literal style
  tabsize=2,                       % sets default tabsize to 2 spaces
  title=\lstname                   % show the filename of files included with \lstinputlisting; also try caption instead of title
}
\usepackage{alltt}
\usepackage[sc]{mathpazo}
\usepackage{amsmath, amsthm, amssymb}
\usepackage{graphicx}
\usepackage[T1]{fontenc}
\usepackage{geometry}
\geometry{verbose,tmargin=2.5cm,bmargin=2.5cm,lmargin=1.5cm,rmargin=1.5cm}
\setcounter{secnumdepth}{2}
\setcounter{tocdepth}{2}
\usepackage{url}
\usepackage[unicode=true,pdfusetitle,
  bookmarks=true,bookmarksnumbered=true,bookmarksopen=true,bookmarksopenlevel=2,
breaklinks=false,pdfborder={0 0 1},backref=false,colorlinks=false]
{hyperref}
\hypersetup{pdfstartview={XYZ null null 1}}
\usepackage{float}
\usepackage{bm}
\usepackage{tikz}
 %changes default sectioning commands 
%\usepackage{breakurl}
\renewcommand{\thesubsection}{(\alph{subsection})}
\renewcommand{\thesubsubsection}{\roman{subsection}.}
\usepackage{lastpage}
\usepackage{fancyhdr}
\pagestyle{fancy}
\lhead{}
\chead{\leftmark}
\rhead{}
\lfoot{Aaron Gonzales; Algorithms}
\cfoot{Homework 1}
\rfoot{Page \thepage\ of \pageref{LastPage}}
\IfFileExists{upquote.sty}{\usepackage{upquote}}{}

\begin{document}

\title{Homework 1, CS561, Fall 2014}
\author{Aaron Gonzales}
\maketitle

%%%%%%%%knitr option%%%%%%%



\section{ exercise 3.1-5 CLRS - Prove theorem 3.1}


\section{ Let $f(n) and g(n)$ be two functions that take on nonnegative values and
assume $f(n) = O(g(n))$. Prove that $g(n) =\Omega(f(n))$.}

By definition, $\Omega(g(n)) = \{ f(n) : $ there exist positive constants $c, n_0$
such that $0 \leq cg(n) \leq f(n) \forall  n \geq n_0\}$.

By definition, $O(g(n)) = \{ f(n) : $ there exist positive constants $c, n_0$
such that $0 \leq f(n) \leq cg(n) \forall  n \geq n_0\}$.

if $f(n)$ is bounded above by $g(n)$, $g(n)$ must be at least equal to or
lower than $f(n)$ for all $n \geq n_0$.





\section{ Problem 3-2 on Relative asymptotic growths}
\begin{table}[h]
  \begin{tabular}{ll|l|l|l|l|l}
	A & B  & $O$  & $o$  & $\Omega$  & $\omega$  & $\Theta$  \\ \hline	
	$lg^k $&  $n^\epsilon$   &  &  &  &  &  \\ \hline	
	$n^k$     & $c^n$  &  &  &  &  &  \\ \hline	
	$\sqrt{n}$ & $n^{sin n}$     &  &  &  &  &  \\ \hline	
	$2^n$      & $2^{n/2}$  &  &  &  &  &  \\ \hline	
	$n^{lg c}$     & $c^{lg n}$  &  &  &  &  &  \\ \hline	
	$lg(n!)$     &  $lg(n^n)$ &  &  &  &  &  
  \end{tabular}
\end{table}


\section{ Assume you have functions $f$ and $g$, such that $f(n)$ is $O(g(n))$.
  for each of the following statements, decide wether you think it is true or false and give
either a proof or a counterexample. }

\subsection{ $log_2f(n) = O(log_2(g(n))$}



\subsection{ $2^{ f(n) } = O(2^{ g(n) })$}

\subsection{$  f(n)^2 = O(g(n)^2)$}


log2(f(n)) is O(log2(g(n))).
Take f(n) = 2 and g(n) = 1. In this case, f(n) is O(g(n)). But, log(1) = 0, and
log(2) is not O(log(1)) i.e. O(0).
However, with the assumption that g(n) is not the constant function 1, and that
g(n) is a non-decreasing function such that g(n) takes values greater than 1
for large
enough n, we can prove that log(f(n)) is O(log(g(n))) as:
f(n) ? C0g(n) 8n > C
Assuming f(n) and g(n) are positive for large enough n, and taking log on both
sides for those values of n, (taking log preserves inequality since log is an
increasing
function)
log(f(n)) ? log(C0g(n)) 8n > C
) log(f(n)) ? log(C0) + log(g(n)) 8n > C
Now, since g(n) is a non-decreasing function and it takes positive values for
large
enough n, let a be the positive integer for which log(g(a)) takes its smallest
positive
value (such an a exists, since g(n) > 1 for large enough n). Now, we wish to
?nd a
constant C00 such that, for large enough n, :
log(C0) ? C00: log(g(n))
i:e: C00 ?
log(C0)
log(g(n))
So, we can choose C00 = log(C0)
log(g(a)) , since log(g(n)) ? log(g(a)) 8n > C000, where
C000 > a is some constant. So, we have:
log(f(n)) ? log(C0) + log\left( g(n)) 8n > C000
? C00 log(g(n)) + log(g(n)) 8n > C000
? (C00 + 1) log(g(n)) 8n > C000
Thus, log(f(n)) is O(log(g(n))) with these assumptions.\right)<++>




\section{ Problem 7-3: Alternative Quicksort analysis}

\section{
  Imagine you are doing a stress test on a particular model of smart phones. you
  have a ladder with $n$ rungs. You want ot determine the highest rung from which
  you cand rop a phone wihtout it breaking and you want to do it iwth the
smallest number of phone drops. }

\subsection{ Imagine that you have exactly 2 phones. Devise an algorithm that
  can determine
the highest safe rung using $o(n)$ drops. (little o). }

  Let it be stated that a ladder must have at least one one rung. If it has
  one rung, we only need one phone to test and see if that rung is safe (if it
  breaks, it's unsafe, if not, it's safe).

  If a ladder has $n > 1$ rungs, let us start by dropping the phone on the
  second rung. If the phone breaks, we drop from the rung below the current
  rung (in this case, the first rung) and if it doesn't break, we know that it
  is safe from the current-1 rung.

  Stated in pseudocode:
  \begin{lstlisting}
	for each rung on ladder:
	  drop phone from rung+1
	  if phone breaks:
		drop phone from rung
	  if phone breaks:
		return current_rung - 2
	  else return rung
  \end{lstlisting}
  we only test at most $n/2+1$ rungs, which dropping the constant is $ n/2 < n
  \forall n \geq 2$. 
  By definition, $o(g(n)) = \{ f(n) : $ for any positive constant $c>0$, there
  exisists a constant $n_0 > 0$ such that $0 \leq f(n) \leq cg(n) \forall n \geq n_0\}$.
  as such, the algorithm is $o(n)$. 

\subsection{Now suppose you have $k$ phones. Devise an algorithm that can
  determine the highest safe rung with the smallest number of drops. If
  $f_k(n)$
  is the number of drops that your algorithm needs, what is 
  $f_k(n)$ asymptotically? Hint: you shoudl ensure that 
  $f_{k+1}(n) = o(f_k(n))$ for any $k$.}

  This is an example of binary search. If we assume that the rungs in the
  ladder are in a sorted array $rungs[]<++>$, the algorithm could be stated
  as follows:

  \begin{lstlisting}

	def binary_search(val, left = 0, right = nil)
	  right = ladder.size - 1 unless right
	  mid = (left + right) / 2
 
	  if left > right
	  	return null
   
	  if val == ladder[mid]
		return mid
	  elif val > ladder[mid]
		binary_search(val, mid + 1, right)
	  else
		binary_search(val, left, mid - 1)

  \end{lstlisting}

  Proof by induction that $f_k(n) = O(log n)$ follows:
  Let $P(n)$ be the assertion that our binary search of the ladder's rungs
  works correctly. 

  Our base case is a ladder of size 1: If the ladder has one rung, then
  $n=1$ and the function returns 1, which is true. 
  If $P(n)$ works for all $n$, then it works for a ladder of size $n+1$. 

  as we have already assumed a sorted ladder,




  \section{ The game of Match. }
  The game of Match is played with a special deck of 27 cards. Each card has
  three attributes: color, shape and number. The possible color values are
  \{red, blue, green\}, the possible shape values are \{square, circle, heart\},
  and the possible number values are \{1, 2, 3\}. Each of the $3 ∗ 3 ∗ 3 = 27$
  possible combinations is represented by a card in the deck. A match is a set
  of 3 cards with the property that for every one of the three attributes,
  either all the cards have the same value for that attribute or they all have
  different values for that attribute. For example, the following three cards
  are a match: (3, red, square), (2, blue, square), (1, green, square).

  \subsection{If we shuffle the deck and turn over three cards, what is the
	  probability that they form a match? Hint: given the first two cards, what
  is the probability that the third forms a match?}

  This is a ``simple'' probability problem. 

  $ 27 * 26 * 25 $ 

  should be $Pr(1/25)$



  \subsection{If we shuffle the deck and turn over n cards where $n ≤ 27$, what
	  is the expected number of matches, where we count each match separately
	  even if they overlap? Note: The cards in a match do not need to be
  adjacent! Is your expression correct for $n = 27?$}






  \section{ Drunken Debutants}
Drunken Debutants: Imagine that there are n debutants, each with her own
porsche. After a late and wild party, each debutante stumbles into a porsche
selected independently and uniformly at random (thus, more than one debutant
may wind up in a porsche). Let X be a random variable giving the number of
debutants that wind up in their own porsche. Use linearity of expectation to
compute the expected value of X. Now use Markov’s inequality, to bound the
probability that X is larger than k for any positive k.







  \section{ Pairs on a circle}
Imagine n points are distributed uniformly at random on the perimeter of a
circle that has circumference 1. Show that the expected number of pairs of
points that are within distance $\Theta(1/n^2)$ of each other is greater than 1. FYI:
this problem has applications in efficient routing in peer-to-peer networks.
Hint: Partition the circle into $n^2/k$ regions of size $k/n^2$ for some constant
k; then use the Birthday paradox to solve for the necessary k.

  \end{document}
